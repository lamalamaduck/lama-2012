\documentclass[a4paper,11pt,draft]{article}

\usepackage[finnish]{babel}
\usepackage[utf8]{inputenc}
\usepackage[margin=1in]{geometry}
\usepackage{amsfonts,amsmath,amssymb,amsthm,enumitem}
\usepackage{pgf}
\usepackage{tikz}
\usetikzlibrary{arrows,automata}

\newtheorem*{claim}{Väite}

\begin{document}

\subsection*{582206 Laskennan mallit, syksy 2012\\
1. Harjoitusten malliratkaisut}

\begin{enumerate}

\item
  \begin{enumerate}
  \item $\emptyset \subseteq \emptyset$ sillä tyhjä joukko on jokaisen
    joukon osajoukko
  \item $\emptyset \notin \emptyset$ sillä tyhjässä joukossa ei ole
    yhtään alkiota
  \item $\emptyset \in \{\emptyset\}$
  \item $\emptyset \subseteq \{\emptyset\}$
  \item $\{a,b\} \in \{a,b,c,\{a,b\}\}$
  \item $\{a,b\} \subseteq \{a,b,\{a,b\}\}$ sillä $a$ ja $b$ kuuluvat
    oikeinpuoleiseen joukkoon

  \item
    \begin{math}
      \begin{aligned}[t]
        \{a,b\} \not\subseteq \mathcal{P}(\{a,b,\{a,b\}\}) =
        \{&\emptyset, \{a\}, \{b\}, \{\{a,b\}\},\\
        &\{a, b\}, \{a, \{a, b\}\},\\
        &\{b, \{a, b\}\}, \{a, b, \{a, b\}\}\}
      \end{aligned}
    \end{math}

  \item $\{\{a,b\}\} \in \mathcal{P}(\{a,b,\{a,b\}\})$
  \item $\{a,b,\{a,b\}\} - \{a,b\}= \{\{a,b\}\} \neq \{a, b\}$
  \end{enumerate}

\item
  \begin{enumerate}
  \item
    \begin{math}
      \begin{aligned}[t]
        (\{1,3,5\} \cup \{3,1\}) \cap \{3,5,7\}
        &= \{1,3,5\} \cap \{3,5,7\} \\
        &= \{3,5\}
      \end{aligned}
    \end{math}

  \item
    \begin{math}
      \begin{aligned}[t]
        \bigcup\{\{3\},\{3,5\},\bigcap\{\{5,7\},\{7,9\}\}\}
        &= \bigcup\{\{3\},\{3,5\},\{5,7\} \cap \{7,9\}\} \\
        &= \bigcup\{\{3\},\{3,5\},\{7\}\} \\
        &= \{3\} \cup \{3,5\} \cup \{7\} \\
        &= \{3,5,7\}
      \end{aligned}
    \end{math}

  \item
    \begin{math}
      \begin{aligned}[t]
        (\{1,2,5\} - \{5,7,9\}) \cup (\{5,7,9\} - \{1,2,5\})
        &= \{1,2\} \cup \{7,9\} \\
        &= \{1,2,7,9\}
      \end{aligned}
    \end{math}

  \item
    \begin{math}
      \begin{aligned}[t]
        \mathcal{P}(\{7,8,9\}) - \mathcal{P}(\{7,9\})
        = \{\{8\}, \{7,8\}, \{8,9\}, \{7,8,9\}\}
      \end{aligned}
    \end{math} \\
    Tulosjoukkoon siis jäävät ne osajoukot joissa esiintyy 8.
  \item
    $\mathcal{P}(\emptyset) = \{\emptyset\}$
  \end{enumerate}

\item
  \begin{enumerate}
  \item
    \begin{math}
      \begin{aligned}[t]
        \{1\} \times \{1,2\} \times \{1,2,3\}
        &= \{(1,1), (1,2)\} \times \{1,2,3\} \\
        &= \{(1,2,1), (1,1,2), (1,1,3),\\
        &\qquad(1,2,1), (1,2,2), (1,2,3)\}
      \end{aligned}
    \end{math}

  \item
    \begin{math}
      \begin{aligned}[t]
        \emptyset \times \{1,2\} = \emptyset
      \end{aligned}
    \end{math}

    $(a,b) \in \emptyset \times \{1,2\} \Rightarrow a \in \emptyset$
    ja koska tyhjässä joukossa ei ole yhtään alkiota, on karteesinen
    tulo tyhjän joukon kanssa aina tyhjä joukko.

  \item
    \begin{math}
      \begin{aligned}[t]
        \mathcal{P}(\{1,2\}) \times \{1,2\}
        &= \{\emptyset, \{1\}, \{2\}, \{1,2\}\} \times \{1,2\} \\
        &= \{(\emptyset, 1), (\emptyset, 2), (\{1\}, 1), (\{1\}, 2),\\
        &\qquad(\{2\}, 1), (\{2\}, 2), (\{1,2\}, 1), (\{1,2\}, 2)\}
      \end{aligned}
    \end{math}

  \item
    $\mathcal{P}(\{\varepsilon\}) = \{\emptyset, \{\varepsilon\}\}$
  \end{enumerate}

\item Ovatko seuraavat väittämät tosia? Selitä miksi jos ovat tai
  eivät ole.
  \begin{enumerate}
  \item
    \begin{claim}
      $\{\varepsilon\}^{*} = \{\varepsilon\}$
    \end{claim}
    \begin{proof}
    \begin{math}
      \begin{aligned}[t]
        \{\varepsilon\}^*
        &= \{w_1w_2 \ldots w_n \mid w_i \in \{\varepsilon\}
           \text{ kaikilla } i \in \{1, \ldots, n\}\}\\
        &= \{w_1w_2 \ldots w_n \mid w_i = \varepsilon
           \text{ kaikilla } i \in \{1, \ldots, n\}\}\\
        &= \{\varepsilon^n \mid n \ge 1\}\\
        &= \{\varepsilon\} \qedhere
      \end{aligned}
    \end{math}
    \end{proof}

\newpage
  \item
    \begin{claim}
      Mielivaltaisella aakkostolla $\Sigma$ ja millä tahansa kielellä
      $L \subseteq \Sigma^*$, $(L^*)^* = L^*$.
    \end{claim}
    \begin{proof}
      \hfill
      \begin{description}
      \item[$L^* \subseteq (L^*)^*$:] \hfill \\
        Olkoon $w \in L^*$. Nyt
        \begin{equation*}
        w \in (L^*)^* = \left\{w_1 \ldots w_n \mid n \ge 0,\forall i
        \in \{1, \ldots, n\}: w_i \in L^* \right\}
        \end{equation*}
        asettamalla $n = 1$ ja $w_1 = w$.
      \item[$(L^*)^* \subseteq L^*$:] \hfill \\
        Olkoon $w \in (L^*)^*$.

        Tällöin $w = w_1w_2 \ldots w_n$ missä $w_i \in L^*$ jokaisella
        $i \in \{1, \ldots, n\}$.

        Olkoon $i \in \{1, \ldots, n\}$.

        Nyt $w_i = w_{i,1}w_{i,2} \ldots w_{i,k_i}$ missä $w_{i,j} \in
        L$ joten
        \begin{align*}
          w &= w_1w_2 \ldots w_n \\
          &= w_{1,1} \ldots w_{1,k_1} \ldots w_{n,k_n} \in L^*
        \end{align*}
        Nyt siis $w \in L^*$ ja siten $(L^*)^* \subseteq L^*$.
      \end{description}

      On siis osoitettu, että $L^* \subseteq (L^*)^*$ ja $(L^*)^*
      \subseteq L^*$, joten $(L^*)^* = L^*$.
    \end{proof}

  \item
    \begin{claim}
      Jos $a \neq b$, niin $\{a,b\}^* = \{a\}^* \circ (\{b\} \circ \{a\}^*)^*$.
    \end{claim}
    \begin{proof}
      Olkoon $w \in \{a,b\}^*$ ja merkitään $A = \{a\}^* \circ
      \left(\{b\} \circ \{a\}^*\right)^*$. Todistetaan, että $w \in A$
      induktiolla merkkijonon pituuden $|w|$ suhteen.

      \begin{description}
      \item[Alkuaskel]
        $|w| = 0$ eli $w = \varepsilon$. Nyt $\varepsilon =
        \varepsilon\varepsilon \in A$.

      \item[Induktioaskel]
        Oletetaan, että $u \in A$ kun $|u| < |w|$.

        \begin{itemize}
        \item
          Jos $w = au$ jollain $u \in \{a,b\}^*$, niin
          induktio-oletuksen nojalla $u \in A$ ja $u = u_1u_2$ missä
          $u_1 \in \{a\}^*$ ja $u_2 \in (\{b\} \circ \{a\}^*)^*$.
          Nyt $au_1 \in \{a\}^*$ ja siten $w = (au_1)u_2 \in A$.

        \item
          Jos taas $w = bu$ jollain $u \in \{a,b\}^*$, on $u = u_1
          \ldots u_n$.

          \begin{itemize}
          \item
            Jos $u_i \neq b$ jokaisella $i \in \{1, \ldots, n\}$,
            niin tällöin $u = a^n$, $bu \in (\{b\} \circ
            \{a\}^*)^*$ ja $w = \varepsilon (bu) \in A$.

          \item
            Jos $u_i = b$ jollain $i \in \{1, \ldots, n\}$, jaetaan
%
            \begin{equation*}
              u = u_1 \ldots u_ju_{j+1} \ldots u_n
            \end{equation*}
%
            missä $u_{j+1}$ on ensimmäinen $b$ merkkijonossa $u$. Nyt
%
            \begin{equation*}
              bu_1 \ldots u_j = ba^j \in (\{b\} \circ \{a\}^*)^*
            \end{equation*}
%
            ja induktio-oletuksen nojalla $u_{j+1} \ldots u_n \in A$. Nyt
            $u_{j+1} \ldots u_n = v_1v_2$ missä
%
            \begin{equation*}
              v_1 \in \{a\}^* \text{ ja }
              v_2 \in \{ \{b\} \circ \{a\}^* \}^* \text{.}
            \end{equation*}
%
            Tällöin $v_1 = a^k$ jollain $k \ge 1$. Kuitenkin $u_{j+1} =
            b$, joten $v_1 = \varepsilon$ ja $u_{j+1} \ldots u_n = v_2 \in
            \{\{b\} \circ \{a\}^*\}^*$. Nyt
%
            \begin{align*}
                          & bu_1 \ldots u_j \in \{b\} \circ \{a\}^* &  \\
              \text{ ja } & u_{j+1} \ldots u_n \in \{\{b\} \circ \{a\}^*\}^*
            \end{align*}
%
            joten $w = \varepsilon (bu) \in A$.

          \end{itemize}
        \end{itemize}
      \end{description}

      Olkoon sitten $w \in A$. Nyt $w = u_1 \ldots u_n$ missä $u_i \in
      \{a, b\}$ kaikilla $i$. Siten $w \in \{a,b\}^*$.

      On siis osoitettu, että $\{a,b\}^* \subseteq A$ ja $A \subseteq
      \{a,b\}^*$ joten joukot ovat samat.
    \end{proof}

  \item
    \begin{claim}
      Jos $\Sigma$ on mielivaltainen aakkosto, $\varepsilon \in L_1 \subseteq
      \Sigma^*$ ja $\varepsilon \in L_2 \subseteq \Sigma^*$, niin $(L_1 \circ
      \Sigma^* \circ L_2)^* = \Sigma^*$.
    \end{claim}

    \begin{proof}
      \begin{description}
      \item[]Merkitään $L = (L_1 \circ \Sigma^* \circ L_2)^*$.

      \item[$L \subseteq \Sigma^*$:]
        \hfill \\
        Olkoon $w \in L$. Nyt $w = l_1vl_2$ jollain $l_1 \in L_1$,
        $v \in \Sigma^*$ ja $l_2 \in L_2$ ja koska

        \begin{align*}
          l_1 \in L_1 \subseteq \Sigma^* &\Rightarrow l_1 \in
          \Sigma^* \\
          l_2 \in L_2 \subseteq \Sigma^* &\Rightarrow l_2 \in
          \Sigma^*
        \end{align*}

        niin $w = l_1vl_2 \in \Sigma^* \circ \Sigma^* \circ \Sigma^*
        = \Sigma^*$. Siis $L \subseteq \Sigma^*$.

      \item[$\Sigma^* \subseteq L$:]
        \hfill \\
        Olkoon $w \in \Sigma^*$. Nyt $w = \varepsilon w \varepsilon$
        ja koska $\varepsilon \in L_1$ ja $\varepsilon \in L_2$,
        niin $w \in L$. Siis $\Sigma^* \subseteq L$.
      \end{description}

      Koska $\Sigma^* \subseteq L$ ja $L \subseteq \Sigma^*$, niin
      $\Sigma^* = L = (L_1 \circ \Sigma^* \circ L_2)^*$.
    \end{proof}

  \item
    \begin{claim}
      Kaikilla kielillä $L$, $\emptyset \circ L = L \circ \emptyset = \emptyset$.
    \end{claim}

    \begin{proof}
      Jos $uv \in \emptyset \circ L$, niin $u \in \emptyset$. Koska
      tyhjässä joukossa ei ole yhtään alkiota, niin myös $\emptyset
      \circ L$ on tyhjä joukko. Vastaavasti tapauksella $L \circ
      \emptyset$. Siis $\emptyset \circ L = L \circ \emptyset =
      \emptyset$.
    \end{proof}
  \end{enumerate}

\item
  Olkoon $\Sigma = \{a,b\}$. Esitä joitakin esimerkkejä
  merkkijonoista, jotka kuuluvat tai eivät kuulu alla määriteltyihin
  joukkoihin.

  \begin{enumerate}
  \item
    $\{w \mid w = uu^Ru \text{ jollakin } u \in \Sigma \circ
    \Sigma\}$

    Joukkoon kuuluvat siis merkkijonot $aaaaaa$, $bbbbbb$, $abbaab$
    ja $baabba$.

  \item
    $\{w \mid ww = www\}$

    Jos $ww = www$, niin $|ww| = |www|$ ja $2|w| = 3|w|$.
    Tämä pätee vain jos $|w| = 0$, joten $w = \varepsilon$. Joukkoon
    kuuluu siis vain tyhjä merkkijono.

  \item
    $\{w \mid uvw = wvu \text{ joillakin } u,v \in \Sigma^*\}$

    Valitaan $u = v = \varepsilon$. Nyt $uvw = w = wvu$ kaikilla
    $w$. Joukkoon kuuluvat siis kaikki mahdolliset merkkijonot.

  \item
    $\{w \mid www = uu \text{ jollakin } u \in \Sigma^*\}$

    Esimerkiksi $ab$ kuuluu joukkoon, sillä $(ab)(ab)(ab) =
    (aba)(aba)$. Toisaalta $abbb$ ei kuulu määriteltyyn joukkoon,
    sillä
    \begin{equation*}
    (abbb)(abbb)(abbb) = (abbbab)(bbabbb)
    \end{equation*}
    mutta
    \begin{equation*}
      abbbab \neq bbabbb
    \end{equation*}
    Tämä esimerkki näyttää että kuuluvuusehdoksi ei riitä pituuden
    parillisuus.
  \end{enumerate}

\newpage
\item
  Milloin yhtälö $L^+ = L^* - \{\varepsilon\}$ on tosi? Tässä $L^+ =
  \{l_1l_2 \ldots l_k \mid k \ge 1 \text{ ja } l_i \in L \text{
    kaikilla } i\}$

  \begin{claim}
    $L^+ = L^* - \{\varepsilon\}$ jos ja vain jos $\varepsilon \notin L$.
  \end{claim}

  \begin{proof}
    Jos $w \in L$, niin $w \in L^+$. Täten jos $\varepsilon \notin
    L^+$, niin $\varepsilon \notin L$. Jos $\varepsilon \notin L$,
    niin ei ole olemassa merkkijonoa $l_1l_2 \ldots l_k =
    \varepsilon$ missä $l_i \in L$ kaikilla $i$. Täten $\varepsilon
    \notin L^+$. Muistetaan lisäksi, että $L^* = L^+ \cup
    \{\varepsilon\}$. Nyt pätee

    \begin{align*}
      \varepsilon \notin L &\Leftrightarrow \varepsilon \notin L^+ \\
      &\Leftrightarrow L^+ = L^+ - \{\varepsilon\} \\
      &\Leftrightarrow L^+ = (L^+ \cup \{\varepsilon\}) -
      \{\varepsilon\} \\
      &\Leftrightarrow L^+ = L^* - \{\varepsilon\}
    \end{align*}

    Siis $\varepsilon \notin L \Leftrightarrow L^+ = L^* - \{\varepsilon\}$.
  \end{proof}

\item
  Etsi seuraavat ehdot täyttävät merkkijonot.

  \begin{enumerate}
  \item Kaksi erillaista viiden mittaista merkkijonoa, joilla
    täsmälleen samat alimerkkijonot lukuunottamatta sanoja itseään.

    Merkkijonoilla $ababa$ ja $babab$ on alimerkkijonot
    $\varepsilon$, $a$, $b$, $ab$, $ba$, $aba$, $bab$, $abab$, ja
    $baba$.
  \item Merkkijono joka koostuu merkeistä $a$ ja $b$ eikä ole kahden
    palindromin ketjutus.

    $abaabb$ on halutunlainen, sillä se ei itsessään ole palindromi,
    ja lisäksi $a(baabb)$, $(ab)aabb$, $aba(abb)$, $(abaa)bb$ ja
    $(abaab)b$ eivät ole kahden palindromin ketjutuksia.
  \item Viiden merkin mittainen merkkijono joka sisältää kaikki
    mahdolliset aakkoston $\{a,b\}$ kahden mittaiset merkkijonot
    alimerkkijonoinaan.

    Kaikki kahden mittaiset merkkijonot aakkostosta $\{a,b\}$ ovat
    $aa$, $bb$, $ab$ ja $ba$. Merkkijono $abbaa$ sisältää nämä
    kaikki.
  \end{enumerate}

\end{enumerate}

\end{document}
