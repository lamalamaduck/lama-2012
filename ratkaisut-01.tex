\documentclass[a4paper,11pt]{article}

\usepackage[finnish]{babel}
\usepackage[utf8]{inputenc}
\usepackage{amsfonts,amsmath,amssymb,amsthm,enumitem}
\usepackage{pgf}
\usepackage{tikz}
\usetikzlibrary{arrows,automata}

\newtheorem*{vaite}{Väite}

\begin{document}

582206 Laskennan mallit, syksy 2012
1. Harjoitusten malliratkaisut

\begin{enumerate}

\item
  \begin{enumerate}
  \item $\emptyset \subseteq \emptyset$ sillä tyhjä joukko on jokaisen
    joukon osajoukko
  \item $\emptyset \notin \emptyset$ sillä tyhjässä joukossa ei ole
    yhtään alkiota
  \item $\emptyset \in \{\emptyset\}$
  \item $\emptyset \subseteq \{\emptyset\}$
  \item $\{a,b\} \in \{a,b,c,\{a,b\}\}$
  \item $\{a,b\} \subseteq \{a,b,\{a,b\}\}$ sillä $a$ ja $b$ kuuluvat
    oikeinpuoleiseen joukkoon
  \item $\{a,b\} \not\subseteq \mathcal{P}(\{a,b,\{a,b\}\}) =
    \{\emptyset, \{a\}, \{b\}, \{\{a,b\}\}, \{a, b\}, \{a, \{a, b\}\},
    \{b, \{a, b\}\}, \{a, b, \{a, b\}\}\}$
  \item $\{\{a,b\}\} \in \mathcal{P}(\{a,b,\{a,b\}\})$
  \item $\underbrace{\{a,b,\{a,b\}\} - \{a,b\}}_{\{\{a,b\}\}} \neq \{a, b\}$
  \end{enumerate}

\item
  \begin{enumerate}
  \item
    \begin{flalign*}
      (\{1,3,5\} \cup \{3,1\}) \cap \{3,5,7\}
      &= \{1,3,5\} \cap \{3,5,7\} \\
      &= \{3,5\}
    \end{flalign*}
  \item
    \begin{flalign*}
      \bigcup\{\{3\},\{3,5\},\bigcap\{\{5,7\},\{7,9\}\}\}
      &= \bigcup\{\{3\},\{3,5\},\{5,7\} \cap \{7,9\}\} \\
      &= \bigcup\{\{3\},\{3,5\},\{7\}\} \\
      &= \{3\} \cup \{3,5\} \cup \{7\} \\
      &= \{3,5,7\}
    \end{flalign*}
  \item
    \begin{flalign*}
      (\{1,2,5\} - \{5,7,9\}) \cup (\{5,7,9\} - \{1,2,5\})
      &= \{1,2\} \cup \{7,9\} \\
      &= \{1,2,7,9\}
    \end{flalign*}
  \item
    \begin{flalign*}
      \mathcal{P}(\{7,8,9\}) - \mathcal{P}(\{7,9\})
      = \{\{8\}, \{7,8\}, \{8,9\}, \{7,8,9\}\}
    \end{flalign*}
    Tulosjoukkoon siis jäävät ne osajoukot joissa esiintyy 8.
  \item
    \begin{flalign*}
      \mathcal{P}(\emptyset) = \{\emptyset\}
    \end{flalign*}
  \end{enumerate}

\item
  \begin{enumerate}
  \item
    \begin{flalign*}
      \{1\} \times \{1,2\} \times \{1,2,3\}
      &= \{(1,1), (1,2)\} \times \{1,2,3\} \\
      &= \{(1,2,1), (1,1,2), (1,1,3), (1,2,1), (1,2,2), (1,2,3)\}
    \end{flalign*}
  \item
    \begin{flalign*}
      \emptyset \times \{1,2\} = \emptyset
    \end{flalign*}
    $(a,b) \in \emptyset \times \{1,2\} \Rightarrow a \in \emptyset$
    ja koska tyhjässä joukossa ei ole yhtään alkiota, on karteesinen
    tulo tyhjän joukon kanssa aina tyhjä joukko.
  \item
    \begin{flalign*}
      \mathcal{P}(\{1,2\}) \times \{1,2\}
      &= \{\emptyset, \{1\}, \{2\}, \{1,2\}\} \times \{1,2\} \\
      &= \left\{(\emptyset, 1), (\emptyset, 2), (\{1\}, 1), (\{1\}, 2),\right.\nonumber\\
      &\qquad\left.{}(\{2\}, 1), (\{2\}, 2), (\{1,2\}, 1), (\{1,2\}, 2)\right\}
    \end{flalign*}
  \item
    \begin{flalign*}
      \mathcal{P}(\{\varepsilon\}) = \{\emptyset, \{\varepsilon\}\}
    \end{flalign*}
  \end{enumerate}

\item Ovatko seuraavat väittämät tosia? Selitä miksi jos ovat tai
  eivät ole.
  \begin{enumerate}

  \item
    \begin{vaite}
      $\{\varepsilon\}^{*} = \{\varepsilon\}$
    \end{vaite}
    \begin{proof}
    \begin{flalign*}
      \{\varepsilon\}^* &= \{w_1w_2 \ldots w_n \mid w_i \in \{\varepsilon\}
      \text{ kaikilla } 1 \le i \le n\} \\
      &= \{w_1w_2 \ldots w_n \mid w_i = \varepsilon
      \text{ kaikilla } 1 \le i \le n \} \\
      &= \{\varepsilon^k \mid k \ge 1\} \\
      &= \{\varepsilon\}
    \end{flalign*}
    \end{proof}

  \item
    \begin{vaite}
      Mielivaltaisella aakkostolla $\Sigma$ ja millä tahansa kielellä
      $L \subseteq \Sigma^*$, $(L^*)^* = L^*$.
    \end{vaite}
    \begin{proof}
      \begin{description}
        \item[]
        \item[$L^* \subseteq (L^*)^*$] \hfill \\
          Olkoon $w \in L^*$. Nyt $w \in (L^*)^* = \{w_1w_2 \ldots w_n \mid n
          \ge 0, w_i \in L^* \text{ kaikilla } 1 \le i \le n\}$ asettamalla
          $n = 1$ ja $w_1 = w$.
        \item[$(L^*)^* \subseteq L^*$] \hfill \\
          Olkoon $w \in (L^*)^*$. Tällöin
          \begin{equation*}
            w = w_1w_2 \ldots w_n
          \end{equation*}
          missä $w_i \in L^*$ jokaisella $1 \le i \le n$. Olkoon $1 \le i
          \le n$. Nyt
          \begin{equation*}
            w_i = w_{i,1}w_{i,2} \ldots w_{i,k_i}
          \end{equation*}
          missä $w_{i,j} \in L$ joten
          \begin{align*}
            w &= w_1w_2 \ldots w_n \\
            &= w_{1,1} \ldots w_{1,k_1} \ldots w_{n,k_n}
          \end{align*}
          Nyt siis $w \in L^*$ ja siten $(L^*)^* \subseteq L^*$.
      \end{description}
      \end{proof}

  \item
    \begin{vaite}
      Jos $a \neq b$, niin $\{a,b\}^* = \{a\}^* \circ (\{b\} \circ \{a\}^*)^*$.
    \end{vaite}
    \begin{proof}
      Olkoon $w \in {a,b}^*$. Todistetaan väite induktiolla
      merkkijonon pituuden $|w|$ suhteen.
      
    \end{proof}

  \item
    \begin{vaite}
      Jos $\Sigma$ on mielivaltainen aakkosto, $\varepsilon \in L_1
      \subseteq \Sigma^*$ ja $L_2 \subseteq \Sigma^*$, niin $(L_1
      \circ \Sigma^* \circ L_2)^* = \Sigma^*$.
    \end{vaite}
    \begin{proof}
      \begin{description}
        \item[]Merkitään $L = (L_1 \circ \Sigma^* \circ L_2)^*$.
        \item[$L \subseteq \Sigma^*$:]
          \hfill \\
          Olkoon $w \in L$. Nyt $w = l_1vl_2$ jollain $l_1 \in L_1$,
          $v \in \Sigma^*$ ja $l_2 \in L_2$ ja koska
          \begin{align*}
            l_1 \in L_1 \subseteq \Sigma^* &\Rightarrow l_1 \in
            \Sigma^* \\
            l_2 \in L_2 \subseteq \Sigma^* &\Rightarrow l_2 \in
            \Sigma^*
          \end{align*}
          niin $w = l_1vl_2 \in \Sigma^* \circ \Sigma^* \circ \Sigma^*
          = \Sigma^*$. Siis $L \subseteq \Sigma^*$.
        \item[$\Sigma^* \subseteq L$:]
          \hfill \\
          Olkoon $w \in \Sigma^*$. Nyt $w = \varepsilon w \varepsilon$
          ja koska $\varepsilon \in L_1$ ja $\varepsilon \in L_2$,
          niin $w \in L$. Siis $\Sigma^* \subseteq L$.
        \item[]Koska $\Sigma^* \subseteq L$ ja $L \subseteq
          \Sigma^*$, niin $\Sigma^* = L = (L_1 \circ \Sigma^* \circ
          L_2)^*$.
      \end{description}
    \end{proof}

  \item
    \begin{vaite}
      Kaikilla kielillä $L$, $\emptyset \circ L = L \circ \emptyset = \emptyset$.
    \end{vaite}
    \begin{proof}
      Jos $uv \in \emptyset \circ L$, niin $u \in \emptyset$. Koska
      tyhjässä joukossa ei ole yhtään alkiota, niin myös $\emptyset
      \circ L$ on tyhjä joukko. Vastaavasti tapauksella $L \circ
      \emptyset$. Siis $\emptyset \circ L = L \circ \emptyset =
      \emptyset$.
    \end{proof}
  \end{enumerate}

\item
  Olkoon $\Sigma = \{a,b\}$. Esitä joitakin esimerkkejä
  merkkijonoista, jotka kuuluvat tai eivät kuulu alla määriteltyihin
  joukkoihin.
  \begin{enumerate}
    \item
      $\{w \mid w = uu^Ru \text{ jollakin } u \in \Sigma \circ
      \Sigma\}$

      Joukkoon kuuluvat siis merkkijonot $aaaaaa$, $bbbbbb$, $abbaab$
      ja $baabba$.
    \item
      $\{w \mid ww = www\}$

      Jos $ww = www$, niin $|ww| = |www|$ ja $2|w| = 3|w|$.
      Tämä pätee vain jos $|w| = 0$, joten $w = \varepsilon$. Joukkoon
      kuuluu siis vain tyhjä merkkijono.
    \item
      $\{w \mid uvw = wvu \text{ joillakin } u,v \in \Sigma^*\}$

      Valitaan $u = v = \varepsilon$. Nyt $uvw = w = wvu$ kaikilla
      $w$. Joukkoon kuuluvat siis kaikki mahdolliset merkkijonot.
    \item
      $\{w \mid www = uu \text{ jollakin } u \in \Sigma^*\}$

      Esimerkiksi $ab$ kuuluu joukkoon, sillä $(ab)(ab)(ab) =
      (aba)(aba)$. Toisaalta $abbb$ ei kuulu määriteltyyn joukkoon,
      sillä $(abbb)(abbb)(abbb) = (abbbab)(bbabbb)$ mutta $abbbab \neq
      bbabbb$. Tämä esimerkki näyttää että kuuluvuusehdoksi ei riitä
      pituuden parillisuus. 
  \end{enumerate}

\item
  Milloin yhtälö $L^+ = L^* - \{\varepsilon\}$ on tosi? Tässä $L^+ =
  \{l_1l_2 \ldots l_k \mid k \ge 1 \text{ ja } l_i \in L \text{
    kaikilla } i\}$
  \begin{vaite}
    $L^+ = L^* - \{\varepsilon\}$ jos ja vain jos $\varepsilon \notin L$.
  \end{vaite}
  \begin{proof}
      Jos $w \in L$, niin $w \in L^+$. Täten jos $\varepsilon \notin
      L^+$, niin $\varepsilon \notin L$. Jos $\varepsilon \notin L$,
      niin ei ole olemassa merkkijonoa $l_1l_2 \ldots l_k =
      \varepsilon$ missä $l_i \in L$ kaikilla $i$. Täten $\varepsilon
      \notin L^+$. Muistetaan lisäksi, että $L^* = L^+ \cup
      \{\varepsilon\}$. Nyt pätee
      \begin{align*}
        \varepsilon \notin L &\Leftrightarrow \varepsilon \notin L^+ \\
        &\Leftrightarrow L^+ = L^+ - \{\varepsilon\} \\
        &\Leftrightarrow L^+ = (L^+ \cup \{\varepsilon\}) -
        \{\varepsilon\} \\
        &\Leftrightarrow L^+ = L^* - \{\varepsilon\}
      \end{align*}
      Siis $\varepsilon \notin L \Leftrightarrow L^+ = L^* - \{\varepsilon\}$.
  \end{proof}

\item
  Etsi seuraavat ehdot täyttävät merkkijonot.
  \begin{enumerate}
    \item Kaksi erillaista viiden mittaista merkkijonoa, joilla
      täsmälleen samat alimerkkijonot lukuunottamatta sanoja itseään.

      Merkkijonoilla $ababa$ ja $babab$ on alimerkkijonot
      $\varepsilon$, $a$, $b$, $ab$, $ba$, $aba$, $bab$, $abab$, ja
      $baba$.
    \item Merkkijono joka koostuu merkeistä $a$ ja $b$ eikä ole kahden
      palindromin ketjutus.

      $abaabb$ on halutunlainen, sillä se ei itsessään ole palindromi,
      ja lisäksi $a(baabb)$, $(ab)aabb$, $aba(abb)$, $(abaa)bb$ ja
      $(abaab)b$ eivät ole kahden palindromin ketjutuksia.
    \item Viiden merkin mittainen merkkijono joka sisältää kaikki
      mahdolliset aakkoston $\{a,b\}$ kahden mittaiset merkkijonot
      alimerkkijonoinaan.

      Kaikki kahden mittaiset merkkijonot aakkostosta $\{a,b\}$ ovat
      $aa$, $bb$, $ab$ ja $ba$. Merkkijono $abbaa$ sisältää nämä
      kaikki.
  \end{enumerate}

\end{enumerate}

\end{document}
