\documentclass[a4paper,11pt]{article}

\usepackage[finnish]{babel}
\usepackage[utf8]{inputenc}
\usepackage[margin=2cm]{geometry}
\usepackage{amsfonts,amsmath,amssymb,amsthm,enumitem}
\usepackage{microtype}
\usepackage{pgf}
\usepackage{tikz}
\usetikzlibrary{arrows,automata}

\setenumerate{listparindent=\parindent}

\newtheorem*{claim}{Väite}

\newcommand{\set}[1]{{\left\{ #1 \right\}}}
\newcommand{\ceil}[1]{{\left\lceil#1\right\rceil}}
\newcommand{\Nat}{\mathbb{N}}

\newenvironment{automata}[1][2.8]%
{\begin{tikzpicture}[->,>=stealth',shorten >=1pt,auto,node distance=#1cm,semithick]}%
{\end{tikzpicture}}

\newenvironment{centerautomata}[1][2.2]%
{\begin{center}\begin{automata}[#1]}%
{\end{automata} \end{center}}

\begin{document}

\subsection*{582206 Laskennan mallit, syksy 2012 \\
  \textmd{7. harjoitusten malliratkaisut \\
    Juhana Laurinharju ja Jani Rahkola}}

\begin{enumerate}
  \item
    Esitä pinoautomaatti seuraaville kielille.
    \begin{enumerate}
      \item
        Kaikki palindromit aakkostosta $\Sigma = \set{a, b, c}$.
      \item
        $\set{a^ib^j \mid 0 \le i \le j}$ missä $\Sigma = \set{a, b, c}$
      \item
        $\set{a^ib^jc^k \mid j = i + k}$ missä $\Sigma = \set{a, b, c}$
      \item
        Kaikki aakkoston $\Sigma = \set{0, 1}$ merkkijonot joissa nollia on
        kaksi kertaa niin paljon kuin ykkösiä.
    \end{enumerate}

  \item
    Tarkastellaan kielioppia
%
    \begin{align*}
      S & \to S+T \mid T \\
      T & \to T*F \mid F \\
      F & \to (S) \mid a
    \end{align*}
%
    Muodosta merkkijonon $s=( a+ a)* a$ jäsennyspuu tämän kieliopin
    mukaisesti.

    Etsi jäsennyspuusta jokin juuresta lehteen johtava polku, jolla sama
    muuttuja esiintyy kahdessa solmussa. Muodosta tämän perusteella
    toistuvuusominaisuuden todistuksen ideaa mukaillen jokin merkkijonon $s$
    jako osiin $s=uvxyz$, joilla merkkijono $uv^ixy^iz$ kuuluu tarkasteltavaan
    kieleen kaikilla $i\in N$.

  \item
    Olkoon $A$ aakkoston $\set{0,1}$ kieli, joka koostuu niistä
    merkkijonoista, joissa on sama määrä nollia ja ykkösiä. Tällä kielellä on
    kontekstiton kielioppi
%
    \begin{equation*}
      S \to SS \mid 0S1 \mid 1S0 \mid \varepsilon
    \end{equation*}
%
    \begin{enumerate}
    \item
      Kielen $A$ eräs toistuvuuspituus on 4. Esitä kieleen $A$ kuuluvalle
      merkkijonolle $s=001101$ kaikki eri tavat jakaa se osiin $s=uvxyz$
      toistuvuusominaisuuden ehdot toteuttavalla tavalla (lause 2.30; Sipser
      Theorem 2.34; tässä siis $p=4$).

    \item
      Onko kielellä $A$ pienempiä toistuvuuspituuksia kuin 4? Perustele.
    \end{enumerate}

  \item
    \begin{enumerate}
      \item
        Koostukoon aakkoston $\set{\mathrm{a},\mathrm{b},\mathrm{c}}$ kieli
        $A$ merkkijonoista, joissa on yhtä monta $\mathrm{a}$-, $\mathrm{b}$-
        ja $\mathrm{c}$-merkkiä. Osoita, että $A$ ei ole yhteydetön.

      \item
        Osoita, että kieli $\set{0^n1^n0^n1^n\mid n\in N}$ ei ole yhteydetön.
    \end{enumerate}

  \item
    Anna yhteydetön kielioppi, joka tuottaa kielen $\set{a^ib^jc^k \mid i = 2j
      \text{ tai } j = 2k}$. Muodosta apulauseen 2.21 mukaisesti kieliopistasi
    pinoautomaatti, joka tunnistaa saman kielen.

  \item
    Tee alla olevasta pinoautomaatista Apulauseen 2.27 mukaisesti kielioppi.

    \begin{centerautomata}
      \node[initial,state,accepting]   (q0)                     {$q_0$};
      \node[state]           (q1) [above right of=q0] {$q_1$};
      \node[state,accepting] (q3) [below right of=q0] {$q_3$};
      \node[state]           (q2) [above right of=q3] {$q_2$};

      \path (q0) edge              node {$\varepsilon, \varepsilon \to \$ $} (q1)
            (q1) edge [loop above] node {$\begin{aligned}
                                           0, & \varepsilon \to  0 \\
                                           1, & \varepsilon \to  1
                                         \end{aligned}$} ()
                 edge node        {$\varepsilon,\varepsilon\to\varepsilon$} (q2)
            (q2) edge [loop right] node {$\begin{aligned}
                                          0, & 0 \to \varepsilon \\
                                          1, & 1 \to \varepsilon
                                          \end{aligned}$} ()
                 edge node        {$\varepsilon,\$ \to \varepsilon$} (q3);
    \end{centerautomata}

\end{enumerate}

\end{document}
