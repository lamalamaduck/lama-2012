\documentclass[a4paper,11pt]{article}

\usepackage[finnish]{babel}
\usepackage[utf8]{inputenc}
\usepackage[margin=2cm]{geometry}
\usepackage{amsfonts,amsmath,amssymb,amsthm,enumitem}
\usepackage{microtype}
\usepackage{pgf}
\usepackage{tikz}
\usetikzlibrary{arrows,automata}

\setenumerate{listparindent=\parindent}

\newtheorem*{claim}{Väite}

\newcommand{\set}[1]{{\left\{ #1 \right\}}}
\newcommand{\ceil}[1]{{\left\lceil#1\right\rceil}}
\newcommand{\Nat}{\mathbb{N}}

\begin{document}

\subsection*{582206 Laskennan mallit, syksy 2012 \\
  \textmd{7. harjoitusten malliratkaisut \\
    Juhana Laurinharju ja Jani Rahkola}}

\begin{enumerate}
  \item
    Esitä pinoautomaatti seuraaville kielille.
    \begin{enumerate}
      \item
        Kaikki palindromit aakkostosta $\Sigma = \set{a, b, c}$.
      \item
        $\set{a^ib^j \mid 0 \le i \le j}$ missä $\Sigma = \set{a, b, c}$
      \item
        $\set{a^ib^jc^k \mid j = i + k}$ missä $\Sigma = \set{a, b, c}$
      \item
        Kaikki aakkoston $\Sigma = \set{0, 1}$ merkkijonot joissa nollia on
        kaksi kertaa niin paljon kuin ykkösiä.
    \end{enumerate}
\end{enumerate}

\end{document}
