\documentclass[a4paper,11pt]{article}

\usepackage[finnish]{babel}
\usepackage[utf8]{inputenc}
\usepackage[margin=2cm]{geometry}
\usepackage{amsfonts,amsmath,amssymb,amsthm,enumitem}
\usepackage{microtype}
\usepackage{pgf}
\usepackage{tikz}
\usetikzlibrary{arrows,automata}

\setenumerate{listparindent=\parindent}

\newtheorem*{claim}{Väite}

\newcommand{\set}[1]{{\left\{ #1 \right\}}}
\newcommand{\ceil}[1]{{\left\lceil#1\right\rceil}}
\newcommand{\Nat}{\mathbb{N}}

\begin{document}

\subsection*{582206 Laskennan mallit, syksy 2012 \\
  \textmd{6. harjoitusten malliratkaisut \\
    Juhana Laurinharju ja Jani Rahkola}}

\begin{enumerate}

  \subsubsection*{Säännölliset kielet}

\item
  Osoita seuraavat kielet epäsäännöllisiksi käyttäen
  pumppauslemmaa (tai jollain muulla haluamallasi tavalla):
  \begin{enumerate}
  \item
    $\set{a^m b^n c^n\mid n,m\geq 1}$
  \item
    aakkoston $\set{a,b, c}$ palindromit
  \item
    $\set{0^n10^n\mid n\in N}$.
  \end{enumerate}

\item
  Mitkä seuraavista kielistä ovat säännöllisiä, mitkä
  eivät (kielillä $A_1$ ja $A_2$ aakkostona $\set{0,1}$, muilla
  $\set{\mathrm{a},\mathrm{b},\mathrm{c}}$):
  \begin{align*}
    A_1 &=\set{0^n1^m0^n\mid n,m\in N}
    &A_2 &=\set{0^n0^n\mid n\in N}
    \\
    A_3 &=\set{ww^\mathcal{R}\mid w\in\Sigma^{\ast}}
    &A_4 &=\set{wuw^\mathcal{R}\mid w,u\in\Sigma^+}.
    \\
    A_5 &=\set{wxw^\mathcal{R}\mid w\in\Sigma^{\ast},x\in\Sigma}
    &A_6 &=\set{\mathrm{abc}\mathrm{a}^n\mathrm{b}^n\mathrm{c}^n\mid n\in N}
  \end{align*}
  Perustele. Voit käyttää hyväksi kaikkia tunnettuja säännöllisiä kieliä
  koskevia ominaisuuksia, etenkin edellisen tehtävän tuloksia.

  \subsubsection*{Kontekstittomat kielet}

\item
  Esitä kontekstittomat kieliopit, jotka tuottavat seuraavat aakkoston
  $\Sigma=\set{0,1}$ kielet:
  \begin{enumerate}
  \item
    parittoman mittaiset merkkijonot
    \begin{align*}
      S & \to MT \\
      T & \to MMT \mid \varepsilon \\
      M & \to 0 \mid 1
    \end{align*}
  \item
    merkkijonot, joilla on osamerkkijono 111
    \begin{align*}
      S & \to H111H \\
      H & \to MH \mid \varepsilon \\
      M & \to 0 \mid 1
    \end{align*}
  \item
    merkkijonot, joissa on ainakin kaksi merkkiä ja joiden
    ensimmäinen ja viimeinen merkki ovat samat
    \begin{align*}
      S & \to 1H1 \mid 0H0 \\
      H & \to MH \mid \varepsilon \\
      M & \to 0 \mid 1
    \end{align*}
  \item
    parittoman mittaiset merkkijonot, joiden ensimmäinen ja
    keskimmäinen merkki ovat samat.
    \begin{align*}
      S & \to 1T_1M \mid 0T_0M \\
      T_1 & \to MT_1M \mid 1 \\
      T_0 & \to MT_0M \mid 0 \\
      M & \to 0 \mid 1
    \end{align*}
  \end{enumerate}

\item
  Esitä kontekstittomat  kieliopit seuraaville kielille:
  \begin{enumerate}
  \item $01^\ast\cup10^\ast$
  \item $\set{0^n1^m\mid\mbox{$m,n\in N$ ja $m\geq n$}}$
  \item $\set{0^n1^k0^m\mid\mbox{$m,n,k\in N$ ja $k=n+m$}}$
  \item $\set{a^n b^m c^m\mid m,n\in N}$
  \item
    aakkoston $\set{0,1}$ merkkijonot, joissa on yhtä paljon nollia ja
    ykkösiä.
  \end{enumerate}


\item
  Täydennä Jyrkin luentojen lauseen 2.3 todistus (s. 140)
  osoittamalla, että kontekstiton kielten luokka on suljettu myös
  konkatenaation ja tähtioperaation suhteen. Esitä todistus samalla
  tarkkuustasolla kuin luentomuistiinpanoissa esitetty yhdisteen
  tapaus.
  
  Olkoon $A$ ja $B$ aakkoston $\Sigma$ yhteydettömiä kieliä ja $A =
  L(G_A)$ ja $B = L(G_B)$ kieliopeilla $G_A = (V_A, \Sigma, R_A, S_A)$
  ja $G_B = (V_B, \Sigma, R_B, S_B)$. Oletetaan $V_A \cap V_B =
  \emptyset$.
%
  \begin{claim}
    Kieli $A \circ B$ on yhteydetön.
  \end{claim}
  \begin{proof}
    Luodaan uusi kielioppi, jossa on uusi lähtösymboli $S \notin V_A
    \cup V_B$, ja sääntö $S \to S_AS_B$ jolla tästä uudesta
    lähtösymbolista voi tuottaa alkuperäisten kielten lähtösymbolien
    katenaation.
    \begin{align*}
      G_{A \circ B} & = (V_{A \circ B}, \Sigma, R_{A \circ B}, S) \\
      V_{A \circ B} & = V_A \cup V_B \cup \set{S} \\
      R_{A \circ B} & = R_A \cup R_B \cup \set{S \to S_AS_B}
    \end{align*}
  \end{proof}
%
  \begin{claim}
    Kieli $A^*$ on yhteydetön.
  \end{claim}
  \begin{proof}
    Luodaan uusi kielioppi, jossa on uusi lähtösymboli $S \notin V_A
    \cup V_B$, ja sääntö $S \to S_AS \mid \varepsilon$ joka
    mahdollistaa $A$:n merkkijonojen toistamisen.
    \begin{align*}
      G_{A^*} & = (V_{A^*}, \Sigma, R_{A^*}, S) \\
      V_{A^*} & = V_A \cup \set{S} \\
      R_{A^*} & = R_A \cup \set{S \to S_AS \mid \varepsilon}
    \end{align*}
  \end{proof}

\item
  Voidaan osoittaa, että kieli $A=\set{ a^n b^n c^n\mid n\in N}$
  ei ole kontekstiton.
  (Tähän palataan myöhemmin kurssilla.)
  Käyttäen tätä tietoa hyväksi osoita, että
  kontekstiton kielten luokka ei ole suljettu leikkauksen suhteen.
  ({\em Vihje:} esitä $A$ kahden kontekstittoman kielen leikkauksena.)
  Päättele edelleen, että kontekstittomien kielten luokka
  ei ole suljettu komplementoinnin suhteen.

\item
  Osoita, että seuraavien aakkoston $\set{a,b,c}$ kielten komplementit ovat kontekstittomia:
  \begin{enumerate}
  \item $A_1=\set{ a^n b^n \mid n \in N}$
  \item $A_2=\set{ a^n b^n c^n\mid n \in N}$
  \end{enumerate}

  Vihje: Voit tietysti yksinkertaisesti kirjoittaa kontekstittomat  kieliopit komplementeille $\overline{A_1}$ ja $\overline{A_2}$. Voi kuitenkin
  olla helpompaa esittää $\overline{A_1}$ ja $\overline{A_2}$ yhdisteinä yksinkertaisemmista kielistä, jotka on suoraviivaisempaa
  nähdä kontekstittomiksi.

  %
  %\item 
  %Todista, 
  %että kieli 
  %$A_2=\set{ 0^n 1^n 0^n 1^n\mid n \in N}$ ei ole kontekstiton.





\end{enumerate}

\end{document}
