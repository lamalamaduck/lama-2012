\documentclass[a4paper,11pt]{article}

\usepackage[finnish]{babel}
\usepackage[utf8]{inputenc}
\usepackage[margin=2cm]{geometry}
\usepackage{amsfonts,amsmath,amssymb,amsthm,enumitem}
\usepackage{microtype}
\usepackage{pgf}
\usepackage{tikz}
\usetikzlibrary{arrows,automata}

\setenumerate{listparindent=\parindent}

\newtheorem*{claim}{Väite}

\newcommand{\set}[1]{{\left\{ #1 \right\}}}
\newcommand{\ceil}[1]{{\left\lceil#1\right\rceil}}
\newcommand{\Nat}{\mathbb{N}}

\begin{document}

\subsection*{582206 Laskennan mallit, syksy 2012 \\
  \textmd{6. harjoitusten malliratkaisut \\
    Juhana Laurinharju ja Jani Rahkola}}

\begin{enumerate}

  \subsubsection*{Säännölliset kielet}

\item
  Osoita seuraavat kielet epäsäännöllisiksi käyttäen
  pumppauslemmaa (tai jollain muulla haluamallasi tavalla):
  \begin{enumerate}
  \item
    \begin{equation*}
      A = \set{a^m b^n c^n\mid n,m\geq 1}
    \end{equation*}

    Ensinnäkin huomataan, että $A$ on säännöllinen täsmälleen silloin, kun
    $A^\mathcal{R}$ on säännöllinen, sillä aikaisemmin ollaan laskuharjoituksissa
    näytetty, että säännölliset kielet ovat suljettu kääntämisen suhteen.
    Riittää siis näyttää kieli $A^\mathcal{R}$ epäsäännölliseksi, missä
    %
    \begin{equation*}
      A^\mathcal{R} = \set{c^nb^na^m \mid n,m \geq 1}.
    \end{equation*}

    Tehdään vastaoletus. Oletetaan, että kieli $A^\mathcal{R}$ on säännöllinen.
    Tällöin sillä on jokin pumppauspituus $p$. Nyt voidaan valita merkkijono
    %
    \begin{equation*}
      s = c^pb^pa \in A^\mathcal{R} \textrm{,}
    \end{equation*}
    %
    jolla pätee $|s| \geq p$. Säännöllisten kielten pumppauslemman nojalla $s$
    voidaan nyt jakaa kolmeen osaan siten, että
    %
    \begin{align*}
      s    & = xyz, \\
      |xy| & \leq p \textrm{ ja}\\
      |y|  & > 0
    \end{align*}
    %
    joten
    \begin{alignat*}{2}
      xy & = c^k         && \textrm{ jollain } k \leq p, \\
      z  & = c^{p-k}b^pa && \textrm{ ja erityisesti} \\
      y  & = c^n         && \textrm{ jollain } n > 0.
    \end{alignat*}

    Nyt pumppauslemman nojalla myös merkkijonon
    %
    \begin{equation*}
      xz = c^{k-n}c^{p-k}b^pa = c^{p-n}b^pa
    \end{equation*}
    %
    tulisi kuulua kieleen $A^\mathcal{R}$. Nyt kuitenkin $n > 0$, joten
    %
    \begin{equation*}
      xz = c^{p-n}b^pa \notin A.
    \end{equation*}

    Tämä on ristiriidassa säännöllisten kielten pumppauslemman kanssa, joten
    kielellä $A^\mathcal{R}$ ei ole pumppausominaisuutta, eikä se siten voi olla
    säännöllinen, joten myöskään kieli $A$ ei ole säännöllinen.

  \item
    aakkoston $\set{a,b,c}$ palindromit
  \item
    \begin{claim}
      Kieli $A = \set{0^n10^n\mid n\in N}$ on säännöllinen.
    \end{claim}
    \begin{proof}
      Oletetaan vastoin, että $A$ on säännöllinen. Nyt on olemassa
      pumppauspituus $p \in \Nat$. Valitaan $A$:n merkkijono
      \begin{equation*}
        s = 0^p10^p
      \end{equation*}
      jolla selvästi $|s| \ge p$. Nyt jos
      \begin{equation*}
        s = xyz \text{, } |xy| \le p \text{ ja }|y| > 0 \text{,}
      \end{equation*}
      niin $xy = 0^k$ jollain $0 < k \le p$ ja erityisesti $y = 0^n$
      jollain $n \ge 1$. Kuitenkin
      \begin{equation*}
        xz = 0^{k-n}0^{p-k}10^p = 0^{p-n}10^p \notin A
      \end{equation*}
      sillä $p - n \neq p$, mikä on ristiriidassa pumppauslemman kanssa.
      Siis $A$ ei voi olla säännöllinen.
    \end{proof}
  \end{enumerate}

\item
  Mitkä seuraavista kielistä ovat säännöllisiä, mitkä
  eivät (kielillä $A_1$ ja $A_2$ aakkostona $\set{0,1}$, muilla
  $\set{\mathrm{a},\mathrm{b},\mathrm{c}}$):
  \begin{align*}
    A_1 &=\set{0^n1^m0^n\mid n,m\in N}
    &A_2 &=\set{0^n0^n\mid n\in N}
    \\
    A_3 &=\set{ww^\mathcal{R}\mid w\in\Sigma^{\ast}}
    &A_4 &=\set{wuw^\mathcal{R}\mid w,u\in\Sigma^+}.
    \\
    A_5 &=\set{wxw^\mathcal{R}\mid w\in\Sigma^{\ast},x\in\Sigma}
    &A_6 &=\set{\mathrm{abc}\mathrm{a}^n\mathrm{b}^n\mathrm{c}^n\mid n\in N}
  \end{align*}
  Perustele. Voit käyttää hyväksi kaikkia tunnettuja säännöllisiä kieliä
  koskevia ominaisuuksia, etenkin edellisen tehtävän tuloksia.

  \begin{description}
    \item[$A_1$] ei ole säännöllinen, sillä merkkijono $s = 0^p10^p$ ei
      pumppaudu.
    \item[$A_2$] on säännöllinen, sillä $L((00)^*) = A_2$.
    \item[$A_3$] ei ole säännöllinen, sillä merkkijono $s = a^pbba^p$ ei
      pumppaudu.
    \item[$A_4$] ei ole säännöllinen, sillä merkkijono $s = a^pba^p$ ei
      pumppaudu.
    \item[$A_5$] ei ole säännöllinen, sillä merkkijono $s = a^pba^p$ ei
      pumppaudu.
    \item[$A_6$] ei ole säännöllinen, sillä $A_6^\mathcal{R}$ ei ole
      säännöllinen. Tämä voidaan nähdä sillä,\\ että \mbox{$s =
        c^pb^pa^pcba \in A_6^\mathcal{R}$} ei pumppaudu.
  \end{description}

  \subsubsection*{Kontekstittomat kielet}

\item
  Esitä kontekstittomat kieliopit, jotka tuottavat seuraavat aakkoston
  $\Sigma=\set{0,1}$ kielet:
  \begin{enumerate}
  \item
    parittoman mittaiset merkkijonot
    \begin{align*}
      S & \to MT \\
      T & \to MMT \mid \varepsilon \\
      M & \to 0 \mid 1
    \end{align*}
  \item
    merkkijonot, joilla on osamerkkijono 111
    \begin{align*}
      S & \to H111H \\
      H & \to MH \mid \varepsilon \\
      M & \to 0 \mid 1
    \end{align*}
  \item
    merkkijonot, joissa on ainakin kaksi merkkiä ja joiden
    ensimmäinen ja viimeinen merkki ovat samat
    \begin{align*}
      S & \to 1H1 \mid 0H0 \\
      H & \to MH \mid \varepsilon \\
      M & \to 0 \mid 1
    \end{align*}
  \item
    parittoman mittaiset merkkijonot, joiden ensimmäinen ja
    keskimmäinen merkki ovat samat.
    \begin{align*}
      S & \to 1T_1M \mid 0T_0M \\
      T_1 & \to MT_1M \mid 1 \\
      T_0 & \to MT_0M \mid 0 \\
      M & \to 0 \mid 1
    \end{align*}
  \end{enumerate}

\item
  Esitä kontekstittomat  kieliopit seuraaville kielille:
  \begin{enumerate}
  \item $01^\ast\cup10^\ast$
    %
    \begin{align*}
      S   & \to 0T_1 \mid 1T_0 \\
      T_1 & \to 1T_1 \mid \varepsilon \\
      T_0 & \to 0T_0 \mid \varepsilon
    \end{align*}
    %
  \item $\set{0^n1^m\mid\mbox{$m,n\in N$ ja $m\geq n$}}$
    %
    \begin{align*}
      S      & \to T_{01} T_1 \\
      T_{01} & \to 0T_{01}1 \mid \varepsilon \\
      T_{1}  & \to 1T_1 \mid \varepsilon
    \end{align*}
    %
  \item $\set{0^n1^k0^m\mid\mbox{$m,n,k\in N$ ja $k=n+m$}}$
    %
    \begin{align*}
      S & \to T_{01} T_{10} \\
      T_{01} & \to 0T_{01}1 \mid \varepsilon \\
      T_{10} & \to 1T_{10}0 \mid \varepsilon
    \end{align*}
    %
  \item $\set{a^n b^m c^m\mid m,n\in N}$
    %
    \begin{align*}
      S & \to AT_{bc} \\
      A & \to aA \mid \varepsilon \\
      T_{bc} & \to bT_{bc}c
    \end{align*}
    %
  \item
    aakkoston $\set{0,1}$ merkkijonot, joissa on yhtä paljon nollia ja
    ykkösiä.
    %
    \begin{align*}
      S & \to 0S1S \mid 1S0S \mid \varepsilon
    \end{align*}
  \end{enumerate}

\item
  Täydennä Jyrkin luentojen lauseen 2.3 todistus (s. 140)
  osoittamalla, että kontekstiton kielten luokka on suljettu myös
  konkatenaation ja tähtioperaation suhteen. Esitä todistus samalla
  tarkkuustasolla kuin luentomuistiinpanoissa esitetty yhdisteen
  tapaus.
  
  Olkoon $A$ ja $B$ aakkoston $\Sigma$ yhteydettömiä kieliä ja $A =
  L(G_A)$ ja $B = L(G_B)$ kieliopeilla $G_A = (V_A, \Sigma, R_A, S_A)$
  ja $G_B = (V_B, \Sigma, R_B, S_B)$. Oletetaan $V_A \cap V_B =
  \emptyset$.
  %
  \begin{claim}
    Kieli $A \circ B$ on yhteydetön.
  \end{claim}
  \begin{proof}
    Luodaan uusi kielioppi, jossa on uusi lähtösymboli $S \notin V_A
    \cup V_B$, ja sääntö $S \to S_AS_B$ jolla tästä uudesta
    lähtösymbolista voi tuottaa alkuperäisten kielten lähtösymbolien
    katenaation.
    \begin{align*}
      G_{A \circ B} & = (V_{A \circ B}, \Sigma, R_{A \circ B}, S) \\
      V_{A \circ B} & = V_A \cup V_B \cup \set{S} \\
      R_{A \circ B} & = R_A \cup R_B \cup \set{S \to S_AS_B}
    \end{align*}
  \end{proof}
  %
  \begin{claim}
    Kieli $A^*$ on yhteydetön.
  \end{claim}
  \begin{proof}
    Luodaan uusi kielioppi, jossa on uusi lähtösymboli $S \notin V_A
    \cup V_B$, ja sääntö $S \to S_AS \mid \varepsilon$ joka
    mahdollistaa $A$:n merkkijonojen toistamisen.
    \begin{align*}
      G_{A^*} & = (V_{A^*}, \Sigma, R_{A^*}, S) \\
      V_{A^*} & = V_A \cup \set{S} \\
      R_{A^*} & = R_A \cup \set{S \to S_AS \mid \varepsilon}
    \end{align*}
  \end{proof}

\item
  Voidaan osoittaa, että kieli $A=\set{ a^n b^n c^n\mid n\in N}$ ei
  ole kontekstiton. (Tähän palataan myöhemmin kurssilla.) Käyttäen
  tätä tietoa hyväksi osoita, että kontekstiton kielten luokka ei ole
  suljettu leikkauksen suhteen. ({\em Vihje:} esitä $A$ kahden
  kontekstittoman kielen leikkauksena.) Päättele edelleen, että
  kontekstittomien kielten luokka ei ole suljettu komplementoinnin
  suhteen.
  %
  \begin{claim}
    Yhteydettömien kielten luokka ei ole suljettu leikkauksen suhteen.
  \end{claim}
  \begin{proof}
    Tehtävässä 3 osoitimme antamalla kieliopin, että kieli $A =
    \set{a^nb^mc^m \mid n,m \in \Nat}$ on yhteydetön. Vastaavasti
    voidaan osoittaa yhteydettömäksi kieli $B = \set{a^nb^nc^m \mid
      n,m \in \Nat}$. Kieli $A$ on siis merkkijonot joissa $b$ ja $c$
    merkkejä on yhtämonta. Vastaavasti kieli $B$ on merkkijonot joissa
    $a$ ja $b$ merkkejä on yhtämonta. Nyt leikkauskieli $A \cap B =
    \set{a^nb^nc^n \mid n \in \Nat}$ josta tiedämmä ettei se ole
    yhteydetön. Siispä yhteydettömien kielten luokka ei ole suljettu
    leikkauksen suhteen.
  \end{proof}
  %
  \begin{claim}
    Yhteydettömien kielten luokka ei ole suljettu komplementin suhteen.
  \end{claim}
  \begin{proof}
    Oletetaan vastoin, että yhteydettömät kielet ovat suljettu
    komplementin suhteen. Olkoon nyt $A$ ja $B$ yhteydettömiä kieliä.
    Tällöin
    \begin{align*}
      A \cup B \text{ yhteydetön }
      & \Rightarrow \overline{(A \cup B)} \text{ yhteydetön} \\
      & \Rightarrow \overline{A} \cap \overline{B} \text{ yhteydetön} \\
      & \Rightarrow A \cap B \text{ yhteydetön}
    \end{align*}
    mikä on ristiriita edellä osoitetun kanssa. Siispä yhteydettömien
    kielten luokka ei voi olla suljettu komplementin suhteen.
  \end{proof}

\item
  Osoita, että seuraavien aakkoston $\set{a,b,c}$ kielten komplementit ovat kontekstittomia:
  \begin{enumerate}
  \item $A_1=\set{ a^n b^n \mid n \in N}$
  \item $A_2=\set{ a^n b^n c^n\mid n \in N}$
  \end{enumerate}

  Vihje: Voit tietysti yksinkertaisesti kirjoittaa kontekstittomat  kieliopit komplementeille $\overline{A_1}$ ja $\overline{A_2}$. Voi kuitenkin
  olla helpompaa esittää $\overline{A_1}$ ja $\overline{A_2}$ yhdisteinä yksinkertaisemmista kielistä, jotka on suoraviivaisempaa
  nähdä kontekstittomiksi.

  %
  %\item 
  %Todista, 
  %että kieli 
  %$A_2=\set{ 0^n 1^n 0^n 1^n\mid n \in N}$ ei ole kontekstiton.





\end{enumerate}

\end{document}
