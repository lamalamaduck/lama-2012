\documentclass[a4paper,11pt]{article}

\usepackage[finnish]{babel}
\usepackage[utf8]{inputenc}
\usepackage[margin=2cm]{geometry}
\usepackage{amsfonts,amsmath,amssymb,amsthm,enumitem}
\usepackage{microtype}
\usepackage{pgf}
\usepackage{tikz}
\usetikzlibrary{arrows,automata}

\setenumerate{listparindent=\parindent}

\newtheorem*{claim}{Väite}

\newcommand{\set}[1]{{\left\{ #1 \right\}}}
\newcommand{\ceil}[1]{{\left\lceil#1\right\rceil}}
\newcommand{\Nat}{\mathbb{N}}

\begin{document}

\subsection*{582206 Laskennan mallit, syksy 2012 \\
  \textmd{6. harjoitusten malliratkaisut \\
    Juhana Laurinharju ja Jani Rahkola}}

\begin{enumerate}

\subsubsection*{Säännölliset kielet}

\item
Osoita seuraavat kielet epäsäännöllisiksi käyttäen
pumppauslemmaa (tai jollain muulla haluamallasi tavalla):
\begin{enumerate}
\item
$\set{a^m b^n c^n\mid n,m\geq 1}$
\item
aakkoston $\set{a,b, c}$ palindromit
\item
$\set{0^n10^n\mid n\in N}$.
\end{enumerate}

\item
Mitkä seuraavista kielistä ovat säännöllisiä, mitkä
eivät (kielillä $A_1$ ja $A_2$ aakkostona $\set{0,1}$, muilla
$\set{\mathrm{a},\mathrm{b},\mathrm{c}}$):
\begin{align*}
A_1 &=\set{0^n1^m0^n\mid n,m\in N}
&A_2 &=\set{0^n0^n\mid n\in N}
\\
A_3 &=\set{ww^\mathcal{R}\mid w\in\Sigma^{\ast}}
&A_4 &=\set{wuw^\mathcal{R}\mid w,u\in\Sigma^+}.
\\
A_5 &=\set{wxw^\mathcal{R}\mid w\in\Sigma^{\ast},x\in\Sigma}
&A_6 &=\set{\mathrm{abc}\mathrm{a}^n\mathrm{b}^n\mathrm{c}^n\mid n\in N}
\end{align*}
Perustele.
Voit käyttää hyväksi kaikkia tunnettuja säännöllisiä
kieliä koskevia ominaisuuksia, etenkin edellisen tehtävän tuloksia.



%
%\item % s07 teht 7.4
%(Sipser Problem 1.63)
%\begin{enumerate}
%\item
%Olkoon $A$ ääretön säännöllinen kieli.
%Osoita, että on olemassa äärettömät säännölliset
%kielet $B_1$ ja $B_2$, joilla $B_1\cup B_2=A$ ja
%$B_1\cap B_2=\emptyset$.
%\item
%\newcommand{\iless}{\subset\!\!\!\subset}
%Merkitään $A\iless B$, jos $A\subset B$ ja $B-A$ on ääretön.
%Olkoot $B$ ja $D$ säännöllisiä kieliä, joilla
%$B\iless D$.
%Osoita, että on olemassa säännöllinen kieli $C$, jolla
%$B\iless C$ ja $C\iless D$.
%\end{enumerate}
%

\subsection*{Kontekstittomat kielet}
\item
Esitä kontekstittomat kieliopit, jotka tuottavat seuraaville
aakkoston $\Sigma=\set{0,1}$ kielille:
\begin{enumerate}
\item
parittoman mittaiset merkkijonot
\item
merkkijonot, joilla on osamerkkijono 111
\item
merkkijonot, joissa on ainakin kaksi merkkiä ja joiden
ensimmäinen ja viimeinen merkki ovat samat
\item
parittoman mittaiset merkkijonot, joiden ensimmäinen ja
keskimmäinen merkki ovat samat.
\end{enumerate}

\item
Esitä kontekstittomat  kieliopit seuraaville kielille:
\begin{enumerate}
\item $01^\ast\cup10^\ast$
  %
  \begin{align*}
    S   & \to 0T_1 \mid 1T_0 \\
    T_1 & \to 1T_1 \mid \varepsilon \\
    T_0 & \to 0T_0 \mid \varepsilon
  \end{align*}
  %
\item $\set{0^n1^m\mid\mbox{$m,n\in N$ ja $m\geq n$}}$
  %
  \begin{align*}
    S      & \to T_{01} T_1 \\
    T_{01} & \to 0T_{01}1 \mid \varepsilon \\
    T_{1}  & \to 1T_1 \mid \varepsilon
  \end{align*}
  %
\item $\set{0^n1^k0^m\mid\mbox{$m,n,k\in N$ ja $k=n+m$}}$
  %
  \begin{align*}
    S & \to T_{01} T_{10} \\
    T_{01} & \to 0T_{01}1 \mid \varepsilon \\
    T_{10} & \to 1T_{10}0 \mid \varepsilon
  \end{align*}
  %
\item $\set{a^n b^m c^m\mid m,n\in N}$
  %
  \begin{align*}
    S & \to AT_{bc} \\
    A & \to aA \mid \varepsilon \\
    T_{bc} & \to bT_{bc}c
  \end{align*}
  %
\item
  aakkoston $\set{0,1}$ merkkijonot, joissa on yhtä paljon nollia ja
  ykkösiä.
  %
  \begin{align*}
    S & \to 0S1S \mid 1S0S \mid \varepsilon
  \end{align*}
\end{enumerate}


\item
Täydennä Jyrkin luentojen lauseen 2.3 todistus (s. 140)
osoittamalla, että kontekstiton kielten luokka on suljettu
myös konkatenaation ja tähtioperaation suhteen.
Esitä todistus samalla tarkkuustasolla kuin luentomuistiinpanoissa
esitetty yhdisteen tapaus.


\item
Voidaan osoittaa, että kieli $A=\set{ a^n b^n c^n\mid n\in N}$
ei ole kontekstiton.
(Tähän palataan myöhemmin kurssilla.)
Käyttäen tätä tietoa hyväksi osoita, että
kontekstiton kielten luokka ei ole suljettu leikkauksen suhteen.
({\em Vihje:} esitä $A$ kahden kontekstittoman kielen leikkauksena.)
Päättele edelleen, että kontekstittomien kielten luokka
ei ole suljettu komplementoinnin suhteen.

%\item 
%Esitä pinoautomatit seuraaville aakkoston $\set{a,b,c}$ kielille:
%\begin{enumerate}
%\item kaikki palindromit
%\item $\set{ a^i b^j \mid 0 \le i \le j}$
%\item $ \set{ a^i b^j c^k \mid j = i + k}$
%\end{enumerate}

%\item 
%Esitä kontekstiton kielioppi, joka tuottaa kielen
%$\set{ a^i b^j c^k \mid i =2j \mbox{ tai } j = 2k}$. Muodosta luennolla esitetyllä
%menetelmällä kieliopistasi pinoautomaatti, joka tunnistaa saman kielen.

\item
Osoita, että seuraavien aakkoston $\set{a,b,c}$ kielten komplementit ovat kontekstittomia:
\begin{enumerate}
\item $A_1=\set{ a^n b^n \mid n \in N}$
\item $A_2=\set{ a^n b^n c^n\mid n \in N}$
\end{enumerate}

Vihje: Voit tietysti yksinkertaisesti kirjoittaa kontekstittomat  kieliopit komplementeille $\overline{A_1}$ ja $\overline{A_2}$. Voi kuitenkin
olla helpompaa esittää $\overline{A_1}$ ja $\overline{A_2}$ yhdisteinä yksinkertaisemmista kielistä, jotka on suoraviivaisempaa
nähdä kontekstittomiksi.

%
%\item 
%Todista, 
%että kieli 
%$A_2=\set{ 0^n 1^n 0^n 1^n\mid n \in N}$ ei ole kontekstiton.





\end{enumerate}

\end{document}
