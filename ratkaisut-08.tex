\documentclass[finnish,11pt,draft]{article}

\usepackage[finnish]{babel}
\usepackage[utf8]{inputenc}
\usepackage[margin=2cm]{geometry}
\usepackage{amsfonts,amsmath,amssymb,amsthm,enumitem}
\usepackage{microtype}
\usepackage{array}
\usepackage{booktabs}
\usepackage{pgf}
\usepackage{tikz}
\usepackage{tikz-qtree}
\usetikzlibrary{arrows,automata,positioning}

\setenumerate{listparindent=\parindent}

\newtheorem*{claim}{Väite}

\newcommand{\set}[1]{{\left\{ #1 \right\}}}
\newcommand{\ceil}[1]{{\left\lceil#1\right\rceil}}
\newcommand{\Nat}{\mathbb{N}}
\newcommand{\ve}{\varepsilon}

\newenvironment{automata}[1][2.8]%
{\begin{tikzpicture}[->,>=stealth',shorten >=1pt,auto,node distance=#1cm,semithick]}%
{\end{tikzpicture}}

\newenvironment{centerautomata}[1][2.8]%
{\begin{center}\begin{automata}[#1]}%
{\end{automata} \end{center}}

\newcolumntype{W}{>{\centering\arraybackslash}m{5cm} }
\newcolumntype{V}{>{\centering\arraybackslash}m{2cm} }

\begin{document}


\subsection*{582206 Laskennan mallit, syksy 2012 \\
  \textmd{8. harjoitusten malliratkaisut \\
    Juhana Laurinharju ja Jani Rahkola}}

Moninauhaisissa koneissa $S$-siirtymä eĺi siirtymä missä ei nauha päätä
siirretä mihinkään on usein hyödyllinen. Koska tämä ominaisuus ei muuta
Turing-koneen kielentunnistusominaisuuksia, voit vapaasti käyttää tätä
ominaisuutta ratkaisuissasi.

\begin{enumerate}
\item
  Tehtävässä tarkastellaan Jyrkin luentojen sivuilla~230 ja 241 esitettyjä
  determinististä ja epädeterminististä Turingin konetta kielelle $\set{ww\mid
    w\in\set{0,1}^\ast}$. Koneiden kaavioesitykset on myös annettu seuraavalla
  sivulla.
\begin{enumerate}
\item
Esitä Jyrkin luentojen sivun 230 deterministisen Turingin koneen
laskenta (ts.\ tilanteiden jono) syötteellä 001001.
\item
Esitä Jyrkin luentojen sivun 241 epädeterministiselle Turingin koneelle
yksi hyväksyvä ja yksi hylkäävä laskenta
syötteellä 001001.
\end{enumerate}

\item
Esitä tilakaaviona deterministinen yksinauhainen
Turingin kone, joka tunnistaa kielen
$\set{a^ib^jc^id^j\mid i,j\in N}$.




\item
Esitä tilakaaviona kaksinauhainen Turingin kone, joka tunnistaa
kielen $\set{a^nb^nc^n\mid n\in N}$.
Sopiva tapa merkitä kaksinauhaisen koneen siirtymä
$\delta(r,a_1,a_2)=(s,b_1,b_2,D_1,D_2)$ on esim.
\newcommand{\twotrans}
{\left[\begin{array}{c}
a_1\rightarrow b_1,D_1\\
a_2\rightarrow b_2,D_2
\end{array}\right]}
\bigskip

%% \begin{center}
%% \input{pics/t11t3.pspdftex}
%% \end{center}

\newcommand{\doubletrans}[6]
{$\left[\begin{array}{c}\trans{#1}{#2}{#3} \\ \trans{#4}{#5}{#6}\end{array}\right]$}

\newcommand{\trans}[3]{#1 \to #2 \text{, } #3}

\newcommand{\spc}{\textvisiblespace}

\begin{centerautomata}
    \node[initial,state] (qstart) {$q_{\text{start}}$};
    \node[state] (qloop1) [above right of=qstart] {$q_1$};
    \node[state] (qloop2) [right=4cm of qloop1] {$q_2$};
    \node[state] (qloop3) [below=2.96cm of qloop2] {$q_3$};
    \node[state,accepting] (qaccept) [below right of=qstart] {$q_{\text{accept}}$};

    \path
    (qstart) edge              node        {\doubletrans{a}   {a}   {S}{\spc}{\spc}{S}} (qloop1)
    (qloop1) edge [loop above] node        {\doubletrans{a}   {a}   {R}{\spc}{a}   {R}} (qloop2)
    (qloop1) edge              node        {\doubletrans{b}   {b}   {S}{\spc}{\spc}{L}} (qloop2)
    (qloop2) edge [loop above] node        {\doubletrans{b}   {b}   {R}{a}   {b}   {L}} (qloop2)
    (qloop2) edge              node        {\doubletrans{c}   {c}   {S}{b}   {b}   {S}} (qloop3)
    (qloop3) edge [loop right] node        {\doubletrans{c}   {c}   {R}{b}   {c}   {R}} (qloop3)
    (qloop3) edge              node [swap] {\doubletrans{\spc}{\spc}{S}{\spc}{\spc}{S}} (qaccept)
    ;
\end{centerautomata}

\item Merkkijono-operaatioita. Olkoon syöteaakkosto $\set{a,b}$
\begin{enumerate}
\item 
Esitä tilakaaviona Turing-kone, mikä siirtää lukupään nauhan loppuun eli syötteen oikealle puolelle.

\item
Esitä tilakaaviona Turing-kone, mikä siirtää lukupään nauhan alkuun eli vasempaan laitaan.

\item
Esitä tilakaaviona Turing-kone, mikä siirtää syötteensä yhdellä paikalla oikealle.

\item
Esitä tilakaaviona Turing-kone, mikä kääntää syötteensä toisin päin.

\end{enumerate}

\item Laskentoa Turing-koneella. Olkoon syöte aakkosto $\set{0,1}$.
\begin{enumerate}
\item
Esitä tilakaaviona Turing-kone, mikä kasvattaa yhdellä syötteenään saamaansa binäärilukua. 
\begin{enumerate}
\item 
Oleta että binääriluvun vähiten merkitsevät bitit ovat nauhan alussa.
\item 
Oleta että binääriluvun vähiten merkitsevät bitit ovat nauhan lopussa.   

\end{enumerate}

\item
Esitä  
tilakaaviona Turing-kone, mikä vähentää yhdellä syötteenään saamaansa binäärilukua. 




\item
Esitä tilakaaviona kolminauhainen Turing-kone, mikä saa kahdellä ensimmäisellä nauhalla yhden binääriluvun kullakin, ja joka laskee kolmannelle nauhalle syötelukujen summan.




 
\end{enumerate}


\item 
Esitä tilakaaviona kolminauhainen  Turing-kone, mikä saa yhdellä nauhalla syötteenä binääriluvun ja kirjoittaa toiselle nauhalle binäärilukua vastaavan määrän kirjainta $a$. Kolmatta nauhaa voit käyttää jos tarvitset (onkohan tää liian vaikea...).

\item
Esitä tilakaaviona kolminauhainen  Turing-kone, 
mikä saa kahdellä ensimmäisellä nauhalla yhden binääriluvun kullakin, ja joka laskee kolmannen nauhan  avulla syötelukujen kertolaskun (onkohan tää liian vaikea...).


%
%\item
%Esitä tilakaaviona epädeterministinen Turingin kone, joka
%tunnistaa aakkoston $\set{0,1,\#}$ kielen
%\[\set{\# w_1\# w_2\#\ldots\# w_n\#\mid
%\mbox{$w_i\in\set{0,1}^\ast$ kaikilla $i$ ja %
%$w_i=w_j$ joillakin $i \neq j$}}.\]
%
%\iffalse
%\item
%\begin{enumerate}
%\item {[Sipser Problem~3.15]}
%Osoita, että Turing-ratkeavien kielten luokka on suljettu
%yhdisteen, leikkauksen ja komplementin suhteen.
%\item {[Sipser Problem~3.16]}
%Osoita, että Turing-tunnistettavien kielten luokka on suljettu
%yhdisteen ja leikkauksen suhteen.
%Miksi edellisen kohdan konstruktio komplementille ei tässä
%toimi?
%\end{enumerate}
%\fi
%
%\item {[Sipser Problem~3.9]}
%Merkintä $k$-PDA tarkoittaa pinoautomaattia, jossa on
%käytettävänä $k$ pinoa.
%Siis 0-PDA on NFA ja 1-PDA on tavallinen PDA.
%Osoita, että
%\begin{enumerate}
%\item
%2-PDA pystyy tunnistamaan kieliä, joita 1-PDA ei pysty mutta
%\item
%minkä tahansa 3-PDA:n tunnistama kieli voidaan tunnistaa
%2-PDA:lla.
%\end{enumerate}
%{\em Vihje:} Simuloi Turingin koneen nauhaa kahdella pinolla.
%Esitä ratkaisun periaate pseudokoodilla tms.\ menemättä
%automaattiformalismin yksityiskohtiin.
%
%\item {[Sipser Exercise 3.14]}
%{\em Jonoautomaatti} on muuten kuin pinoautomaatti,
%mutta pino on korvattu jonolla.
%Jonoon voidaan kohdistaa kahdenlaisia operaatioita:
%\begin{itemize}
%\item
%$\mbox{\sc Enqueue}(a)$ kirjoittaa merkin $a$ jonon loppuun ja
%\item
%$\mbox{\sc Dequeue}$ poistaa jonon ensimmäisen merkin ja
%palauttaa sen arvonaan.
%\end{itemize}
%Pinoautomaatin tapaan syöte on luettavissa merkki kerrallaan.
%Sovitaan, että syötteessä on aina loppumerkkinä
%(mutta ei muualla) tyhjämerkki $\square$.
%Turingin koneen tapaan pinoautomaatti
%hyväksyy syötteen siirtymällä erilliseen hyväksyvään tilaan.
%
%Osoita, että mikä tahansa Turing-tunnistettava kieli voidaan
%tunnistaa deterministisellä jonoautomaatilla.
%Perusteluksi riittää esittää sopivan tasoisena
%pseudokoodina, miten Turingin konetta voidaan simuloida
%jonoa käyttäen.
%

\end{enumerate}
\vfill

\centerline{\bf Tehtävän~1 kaaviot seuraavalla sivulla!}

\newpage

%% Tehtävän 1.(a) deterministinen Turingin kone:
%% \begin{center}
%% \input{pics/TMesim02.pspdftex}
%% \end{center}
%% Tehtävän 1.(b) epädeterministinen Turingin kone:
%% \begin{center}
%% \input{pics/NDTMesim01.pspdftex}
%% \end{center}


\end{document}
