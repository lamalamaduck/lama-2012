\documentclass[finnish,11pt,draft]{article}

\usepackage[finnish]{babel}
\usepackage[utf8]{inputenc}
\usepackage[margin=2cm]{geometry}
\usepackage{amsfonts,amsmath,amssymb,amsthm,enumitem}
\usepackage{microtype}
\usepackage{array}
\usepackage{booktabs}
\usepackage{pgf}
\usepackage{tikz}
\usepackage{tikz-qtree}
\usepackage{wasysym}
\usetikzlibrary{arrows,automata,positioning}

\setenumerate{listparindent=\parindent}

\newtheorem*{claim}{Väite}

\newcommand{\set}[1]{{\left\{ #1 \right\}}}
\newcommand{\ceil}[1]{{\left\lceil#1\right\rceil}}
\newcommand{\Nat}{\mathbb{N}}
\newcommand{\ve}{\varepsilon}

\newenvironment{automata}[1][2.8]%
{\begin{tikzpicture}[->,>=stealth',shorten >=1pt,auto,node distance=#1cm,semithick]}%
{\end{tikzpicture}}

\newenvironment{centerautomata}[1][2.8]%
{\begin{center}\begin{automata}[#1]}%
{\end{automata} \end{center}}

\newcommand{\doubletrans}[6]
{$\left[\begin{array}{c}\trans{#1}{#2}{#3} \\ \trans{#4}{#5}{#6}\end{array}\right]$}

\newcommand{\trans}[3]{#1 \to #2 \text{, } #3}

\newcommand{\spc}{\text{\textvisiblespace}}


\newcolumntype{W}{>{\centering\arraybackslash}m{5cm} }
\newcolumntype{V}{>{\centering\arraybackslash}m{2cm} }

\begin{document}

\subsection*{582206 Laskennan mallit, syksy 2012 \\
  \textmd{8. harjoitusten malliratkaisut \\
    Juhana Laurinharju ja Jani Rahkola}}

Moninauhaisissa koneissa $S$-siirtymä eĺi siirtymä missä ei nauha päätä
siirretä mihinkään on usein hyö\-dyl\-li\-nen. Koska tämä ominaisuus ei muuta
Turing-koneen kielentunnistusominaisuuksia, voit vapaasti käyttää tätä
ominaisuutta ratkaisuissasi.

\begin{enumerate}
\item
  Tehtävässä tarkastellaan Jyrkin luentojen sivuilla~230 ja 241 esitettyjä
  determinististä ja e\-pä\-de\-ter\-mi\-nis\-tis\-tä Turingin konetta
  kielelle $\set{ww\mid w\in\set{0,1}^\ast}$. Koneiden kaavioesitykset on myös
  annettu seuraavalla sivulla.
  \begin{enumerate}
  \item
    Esitä Jyrkin luentojen sivun 230 deterministisen Turingin koneen laskenta
    (ts.\ tilanteiden jono) syötteellä 001001.
  \item
    Esitä Jyrkin luentojen sivun 241 epädeterministiselle Turingin koneelle
    yksi hyväksyvä ja yksi hylkäävä laskenta syötteellä 001001.
  \end{enumerate}
  
\item
  Esitä tilakaaviona kielen $\set{a^ib^jc^id^j\mid i,j\in \Nat}$ tunnistava
  deterministinen yksinauhainen Turingin kone.

  \begin{centerautomata}
    \node[initial,state] (q0)                      {$q_0$};
    \node[state]         (q1)  [above left of=q0]  {$q_1$};
    \node[state]         (q2)  [above right of=q0] {$q_2$};
    \node[state]         (q3)  [below=1.4cm of q0] {$q_3$};
    \node[state]         (q4)  [below right of=q3] {$q_4$};
    \node[state]         (q5)  [below left of=q3]  {$q_5$};
    \node[state]         (q6)  [right=2.8cm of q3] {$q_6$};
    \node[accepting,state] (a) [right of=q6]       {$q_{accept}$};

    \path
    (q0) edge              node {$a \to \bar{a},R$}   (q1)
    (q1) edge [loop above] node {$a,b,\bar{c} \to R$} (q1)
    (q1) edge              node {$c \to \bar{c}, L$}  (q2)
    (q2) edge [loop above] node {$a,b,\bar{c} \to L$} (q2)
    (q2) edge              node {$\bar{a} \to R$}     (q0);
    \path
    (q0) edge node {$\spc,b \to S$} (q3);
    \path
    (q3) edge              node {$b \to \bar{b}, R$}        (q4)
    (q4) edge [loop below] node {$b,\bar{c},\bar{d} \to R$} (q4)
    (q4) edge              node {$d \to \bar{d}, L$}        (q5)
    (q5) edge [loop below] node {$b,\bar{c},\bar{d} \to L$} (q5)
    (q5) edge              node {$\bar{b} \to R$}           (q3);
    \path
    (q3) edge              node {$\spc,\bar{c},\bar{d} \to R$} (q6)
    (q6) edge [loop above] node {$\bar{c},\bar{d} \to R$}    (q6)
    (q6) edge              node {$\spc \to S$}                 (a);
  \end{centerautomata}

  Automaatti koostuu kolmesta osasta. Tilasta $q_0$ lähtevässä silmukassa
  tarkistetaan, että $c$ merkkejä on vähintään yhtä monta kuin $a$ merkkejä.
  Tilasta $q_3$ lähtevässä silmukassa puolestaan tarkistetaan, että $d$
  merkkejä on vähintään yhtä monta kuin $b$ merkkejä. Tämä tehdään
  merkitsemällä käsitelty $a$-merkki (vastaavasti $b$-merkki) viivalla,
  etsimällä seuraava merkitsemätön $c$-merkki (vastaavasti $d$-merkki) ja
  merkitsemällä sekin. Jos näin tehtäessä $c$ tai $d$ merkit loppuivat kesken,
  siirrytään virhetilaan (jota ei selvyyden vuoksi piirretty kuvaan).

  Lopuksi tarkistetaan, että yhtään $c$ tai $d$ merkkiä ei jäänyt
  merkkaamatta. Tämä tehdään aloittamalla viimeisestä $b$-merkkiä seuraavasta
  merkistä, ja sallimalla vain merkattuja $c$ ja $d$ merkkejä. Kun koko syöte
  on luettu, siirrytään hyväksyvään tilaan.

\item
  Esitä tilakaaviona kaksinauhainen Turingin kone, joka tunnistaa kielen
  $\set{a^nb^nc^n\mid n\in N}$. Sopiva tapa merkitä kaksinauhaisen koneen
  siirtymä $\delta(r,a_1,a_2)=(s,b_1,b_2,D_1,D_2)$ on esim.

  \newcommand{\twotrans}
{\left[\begin{array}{c}
a_1\rightarrow b_1,D_1\\
a_2\rightarrow b_2,D_2
\end{array}\right]}

\bigskip

%% \begin{center}
%% \input{pics/t11t3.pspdftex}
%% \end{center}

\begin{centerautomata}
    \node[initial,state] (qstart) {$q_{\text{start}}$};
    \node[state] (qloop1) [above right of=qstart] {$q_1$};
    \node[state] (qloop2) [right=4cm of qloop1] {$q_2$};
    \node[state] (qloop3) [below=2.968cm of qloop2] {$q_3$};
    \node[state,accepting] (qaccept) [below right of=qstart] {$q_{\text{accept}}$};

    \path
    (qstart) edge              node        {\doubletrans{a}   {a}   {S}{\spc}{\spc}{S}} (qloop1)
    (qstart) edge              node [swap] {\doubletrans{\spc}{\spc}{S}{\spc}{\spc}{S}} (qaccept)
    (qloop1) edge [loop above] node        {\doubletrans{a}   {a}   {R}{\spc}{a}   {R}} (qloop2)
    (qloop1) edge              node        {\doubletrans{b}   {b}   {S}{\spc}{\spc}{L}} (qloop2)
    (qloop2) edge [loop above] node        {\doubletrans{b}   {b}   {R}{a}   {b}   {L}} (qloop2)
    (qloop2) edge              node        {\doubletrans{c}   {c}   {S}{b}   {b}   {S}} (qloop3)
    (qloop3) edge [loop below] node        {\doubletrans{c}   {c}   {R}{b}   {c}   {R}} (qloop3)
    (qloop3) edge              node [swap] {\doubletrans{\spc}{\spc}{S}{\spc}{\spc}{S}} (qaccept)
    ;
\end{centerautomata}

Idea tässä Turingin koneessa on seuraava:
\begin{enumerate}[label=\arabic*)]
    \item kopioi kaikki $a$-merkit toiselle nauhalle
    \item Jokaisen $b$-merkin kohdalla korvataan toisella nauhalla yksi
        $a$-merkki $b$-merkillä. Tämä tehdään toisella nauhalla lopusta alkuun
        päin.
    \item Jos toisella nauhalla on vielä $a$-merkkejä jäljellä, hylätään.
    \item Jokaisen $c$-merkin kohdalla korvataan toisen nauhan $b$-merkki
        $c$-merkillä. Nyt toisella nauhalla liikutaan taas alusta loppuun päin.
    \item Jos toisella nauhalla on nyt viimeinen $b$ korvattu $c$:llä, niin
        hyväksytään. Muuten hylätään.
\end{enumerate}

\item
  Merkkijono-operaatioita. Olkoon syöteaakkosto $\set{a,b}$
  \begin{enumerate}
  \item 
    Esitä tilakaaviona Turing-kone, mikä siirtää lukupään nauhan loppuun eli
    syötteen oikealle puolelle.
%
    \begin{centerautomata}
      \node[initial,state]   (q0)               {$q_0$};
      \node[accepting,state] (q1) [right of=q0] {$q_1$};

      \path
      (q0) edge [loop above] node {$a,b \to R$} (q0)
      (q0) edge              node {$\spc \to S$}  (q1);
    \end{centerautomata}
  \item
    Esitä tilakaaviona Turing-kone, mikä siirtää lukupään nauhan alkuun eli
    vasempaan laitaan.
%
    \begin{centerautomata}
      \node[initial,state]   (q0)               {$q_0$};
      \node[state]           (q1) [right of=q0] {$q_1$};
      \node[state]           (q2) [below of=q1] {$q_2$};
      \node[accepting,state] (q3) [right of=q1] {$q_3$};

      \path
      (q0) edge node {$\begin{aligned}
                         a & \to \bar{a}, L \\
                         b & \to \bar{b}, L \\
                         \spc & \to \bar{\spc}, L
                       \end{aligned}$} (q1)
      (q1) edge node {$\begin{aligned}
                         a & \to R \\
                         b & \to R \\
                         \spc & \to R
                       \end{aligned}$} (q2)
      (q2) edge [bend left] node {$\begin{aligned}
                                     \bar{a} & \to a, L \\
                                     \bar{b} & \to b, L \\
                                     \bar{\spc} & \to \spc, L
                                   \end{aligned}$} (q0)
      (q1) edge node {$\begin{aligned}
                         \bar{a} & \to a, S \\
                         \bar{b} & \to b, S \\
                         \bar{\spc} & \to \spc, S
                       \end{aligned}$} (q3);
    \end{centerautomata}
  \item
    Esitä tilakaaviona Turing-kone, mikä siirtää syötteensä yhdellä paikalla
    oikealle.
%
    \begin{centerautomata}
      \node[initial right,state] (q0)                                   {$q_0$};
      \node[state]               (qa) [above right=1.4cm and 4cm of q0] {$q_a$};
      \node[state]               (qb) [below right=1.4cm and 4cm of q0] {$q_b$};
      \node[accepting,state]     (a)  [left of=q0]                      {$q_{accept}$};

      \path
      (q0) edge              node        {$a \to \spc,R$}  (qa)
      (q0) edge              node        {$\spc \to R$}    (a)
      (q0) edge              node [swap] {$b \to \spc,R$}  (qb)
      (qa) edge [loop above] node        {$a \to a, R$}  (qa)
      (qa) edge [bend left]  node        {$b \to a, R$}  (qb)
      (qa) edge [bend right] node [swap] {$\spc \to a, R$} (a)
      (qb) edge [bend left]  node        {$a \to b, R$}  (qa)
      (qb) edge [loop below] node        {$b \to b, R$}  (qb)
      (qb) edge [bend left]  node        {$\spc \to b, R$} (a);
    \end{centerautomata}
  \item
    Esitä tilakaaviona Turing-kone, mikä kääntää syötteensä toisin päin.
    \begin{centerautomata}
      \node[initial,state] (q0) {$q_0$};
      \node[state]         (q1) [below of=q0] {$q_1$};
      \node[state]         (q2) [below of=q1] {$q_2$};
      \node[state]         (q3) [below of=q2] {$q_3$};
      \node[accepting,state] (a) [below of=q3] {$q_{accept}$};

      \path
      (q0) edge [loop above] node {\doubletrans{a,b}{a,b}{R}{\spc}{\#}{R}} (q0)
      (q0) edge              node {\doubletrans{\spc}{\spc}{S}{\spc}{\spc}{S}} (q1)
      (q1) edge              node {siirrä nauhan 2 lukupää alkuun} (q2)
      (q2) edge              node {\doubletrans{\spc}{\spc}{L}{\#}{\#}{S}} (q3)
      (q3) edge [loop left]  node {\doubletrans{a}{a}{L}{\#}{a}{R}} (q3)
      (q3) edge [loop right] node {\doubletrans{b}{b}{L}{\#}{b}{R}} (q3)
      (q3) edge              node {\doubletrans{a,b}{a,b}{S}{\spc}{\spc}{S}} (a);
    \end{centerautomata}

    Aluksi siirrytään syötteen sisältävällä ensimmäisellä nauhalla syötteen
    jälkeiseen tyhjään merkkiin. Samalla kirjoitetaan toiselle nauhalle
    syötteen pituutta vastaava määrä $\#$ merkkejä. Seuraavaksi siirretään
    toisen nauhan lukupää nauhan alkuun ja tämän jälkeen kopioidaan syöte
    ensimmäiseltä nauhalta lopusta alkuun toiselle nauhalle. Tämä kääntää
    syötteen. Toiselle nauhalle kirjoitetut $\#$-merkit kertovat milloin
    kopiointi lopetetaan.
  \end{enumerate}

\item
  Laskentoa Turing-koneella. Olkoon syöte aakkosto $\set{0,1}$.
  \begin{enumerate}
  \item
    Esitä tilakaaviona Turing-kone, mikä kasvattaa yhdellä syötteenään
    saamaansa binäärilukua.

    Binäärilukua voi kasvattaa yhdellä seuraavasti:
    \begin{enumerate}[label=\arabic*)]
        \item lähde liikkeelle vähiten merkitsevästä bitistä
        \item muuta jokainen matkalla vastaan tuleva ykkönen nollaksi
        \item muuta ensimmäinen nolla ykköseksi ja lopeta
    \end{enumerate}

    \begin{enumerate}
    \item
      Oleta että binääriluvun vähiten merkitsevät bitit ovat nauhan alussa.
%
    \begin{centerautomata}[3.5]
        \node[initial,state] (start) {$q_{\text{start}}$};
        \node[state,accepting] (accept) [right of=start] {$q_{\text{accept}}$};

        \path
        (start) edge [loop above] node {$1 \to 0, R$} (start)
        (start) edge node {$0,\spc \to 1, R$} (accept);
    \end{centerautomata}

    Automaatti noudattaa yllä kuvattua ideaa. Vähiten merkitsevät ykköset
    muutetaan nolliksi kunnes tulee vastaan nolla, joka muutetaan ykköseksi.
    Jos luvussa ei ole nollia, lisätään loppuun yksi ykkönen.

    \item
      Oleta että binääriluvun vähiten merkitsevät bitit ovat nauhan lopussa.

      Ensimmäinen idea olisi kelata nauha loppuun ja toimia sen jälkeen kuten
      äsken, mutta päinvastaiseen suuntaan.

      \begin{centerautomata}
          \node[initial,state] (start) {$q_{\text{start}}$};
          \node[state] (q1) [right of=start] {$q_1$};
          \node[state,accepting] (accept) [right of=q1] {$q_{\text{accept}}$};

          \path
          (start) edge [loop above] node {$0,1 \to R$} (start)
          (start) edge node {$\spc \to L$} (q1)
          (q1) edge [loop above] node {$1 \to 0, L$} (q1)
          (q1) edge node {$0 \to 1, L$} (accept);
      \end{centerautomata}

      Tässä tulee kuitenkin ongelmaksi tapaus, jossa luvussa on pelkkiä ykkösiä,
      jolloin lukua pitäisi pidentää.

      Korjataan Turingin konetta seuraavasti. Siirretään ensin kaikkia alkioita
      oikealle yhden askeleen verran ja jätetään nauhan alkuun tyhjä merkki.
      Tämän jälkeen voidaan toimia kuten äsken ja tarvittaessa siirtää lopuksi
      alkioita nauhalla yhden askeleen vasemmalle.

      \begin{centerautomata}[3]
          \node[initial,state] (r0) {$r_0$};
          \node[state] (r2) [below left of=r0] {$r_2$};
          \node[state] (r1) [below right of=r0] {$r_1$};

          \node[state] (s1) [below=4.5cm of r0] {$s_1$};
          \node[state] (s2) [right of=s1] {$s_2$};

          \node[state] (t0) [below=2.2cm of s1] {$t_0$};
          \node[state] (t1) [right of=t0] {$t_1$};

          \node[state] (u0) [below=3cm of t1] {$u_0$};
          \node[state] (u1) [left of=u0] {$u_1$};
          \node[state] (u2) [below of=u1] {$u_2$};

          \node[accepting,state] (accept) [left of=t0] {$q_{\text{accept}}$};

          \path
          (r0) edge node {$\trans{0}{\spc}{R}$} (r1)
          (r0) edge node [swap] {$\trans{1}{\spc}{R}$} (r2)
          (r1) edge [loop right] node {$\trans{0}{0}{R}$} (r1)
          (r1) edge [bend left] node {$\trans{1}{0}{R}$} (r2)
          (r2) edge [loop left] node {$\trans{1}{1}{R}$} (r2)
          (r2) edge [bend left] node [swap] {$\trans{0}{1}{R}$} (r1)

          (r2) edge [bend right] node [swap] {$\trans{\spc}{1}{S}$} (s1)
          (r1) edge [bend left] node  {$\trans{\spc}{0}{S}$} (s1)

          (s1) edge [loop below] node {$1 \to 0, L$} (s1)
          (s1) edge node [swap] {$0 \to 1, L$} (s2)

          (s2) edge node {kelaa nauha alkuun} (t0)
          (t0) edge node {$0,1 \to S$} (accept)
          (t0) edge node {$\trans{\spc}{\spc}{S}$} (t1)
          (t1) edge [bend left] node {kelaa nauha loppuun} (u0)

          (u0) edge node [swap] {$\trans{0}{\spc}{L}$} (u1)
          (u0) edge [bend left] node {$\trans{1}{\spc}{L}$} (u2)
          (u1) edge [bend right] node [swap] {$\trans{1}{0}{L}$} (u2)
          (u1) edge [loop above] node {$\trans{0}{0}{L}$} (u1)
          (u2) edge [bend right] node [swap] {$\trans{0}{1}{L}$} (u1)
          (u2) edge [loop below] node {$\trans{1}{1}{L}$} (u2)

          (u1) edge node {$\trans{\spc}{0}{S}$} (accept)
          (u2) edge [bend left=50] node {$\trans{\spc}{1}{S}$} (accept)
          ;
      \end{centerautomata}

      Lamatonttu toivottaa hyvää joulua. \smiley

    \end{enumerate}
  \item
    Esitä tilakaaviona Turing-kone, mikä vähentää yhdellä syötteenään
    saamaansa binäärilukua.

    Vähiten merkitsevä bitti vasemmalla.

    \begin{centerautomata}
        \node[initial,state]   (start)                   {$q_{\text{start}}$};
        \node[state]           (q0)     [right of=start] {$q_0$};
        \node[state]           (q1)     [right of=q0]    {$q_1$};
        \node[state,accepting] (accept) [right of=q1]    {$q_{\text{accept}}$};

        \path
        (start) edge [loop above] node {$\trans{0}{1}{R}$}    (start)
        (start) edge              node {$\trans{1}{0}{R}$}    (q0)
        (q0)    edge [loop above] node {$0,1 \to R$}          (q0)
        (q0)    edge              node {$\spc \to L$}         (q1)
        (q1)    edge [loop above] node {$\trans{0}{\spc}{L}$} (q1)
        (q1)    edge              node {$1,\spc \to S$}       (accept);
    \end{centerautomata}

  \item
    Esitä tilakaaviona kolminauhainen Turing-kone, mikä saa kahdellä
    ensimmäisellä nauhalla yhden binääriluvun kullakin, ja joka laskee
    kolmannelle nauhalle syötelukujen summan.

    Tehdään ensin a- ja b-kohtia käyttäen kaksinauhainen Turingin kone, joka
    laskee summan ensimmäiselle nauhalle. Idea on seuraava:
    \begin{enumerate}[label=\arabic*)]
        \item Jos toisella nauhalla on 0, lopeta. Muuten toista seuraavaa:
        \item Vähennä toisen nauhan lukua yhdellä ja kasvata ensimmäisen nauhan
            lukue yhdellä.
    \end{enumerate}

    \newcommand{\tripletrans}[3]
    {$\left[\begin{array}{c} #1 \\ #2 \\ #3 \end{array}\right]$}

    \newcommand{\doubletranss}[2]
    {$\left[\begin{array}{c} #1 \\ #2 \end{array}\right]$}
%
    \begin{centerautomata}
        \node[initial right,state] (start) {$q_{\text{start}}$};
        \node[state] (q1) [above right of=start] {$q_1$};
        \node[state] (q2) [below right of=q1]    {$q_2$};
        \node[state] (q3) [below right of=start] {$q_3$};
        \node[state,accepting] (accept) [left=4cm of start] {$q_{\text{accept}}$};

        \path
        (start) edge node [swap] {\doubletranss{0,1,\spc \to S}{\spc \to S}} (accept)
        (start) edge [bend left] node {\doubletranss{$0,1,\spc \to S$}{$0,1 \to S$}} (q1)
        (q1) edge [bend left, text width=4cm] node {kasvata ensimmäisen nauhan lukua yhdellä} (q2)
        (q2) edge [bend left, text width=4cm] node {vähennä toisen nauhan luvusta 1} (q3)
        (q3) edge [bend left, text width=4cm] node {kelaa nauhat alkuun} (start);
    \end{centerautomata}

    Vaihtoehtoisesti tämän voi tehdä kolmella nauhalla tavallisella allekainlaskumenetelmällä.

    \begin{centerautomata}[1.2]
        \node[initial above,state] (q0) {$q_0$};
        \node[state] (q1) [right=2.5cm of q0] {$q_1$};
        \node[state,accepting] (accept) [below right=4.5cm and 0.95cm of q0] {$q_{\text{accept}}$};

        \path
        (q0) edge [loop left]  node {\begin{tabular}{rl}
                                       \tripletrans{0, \spc \to R}
                                                   {0, \to R}
                                                   {\trans{\spc}{0}{R}} &
                                       \tripletrans{0 \to R}
                                                   {\spc, \to R}
                                                   {\trans{\spc}{0}{R}} \\
                                       \tripletrans{1 \to R}
                                                   {0,\spc \to R}
                                                   {\trans{\spc}{1}{R}} &
                                       \tripletrans{0,\spc \to R}
                                                   {1 \to R}
                                                   {\trans{\spc}{1}{R}}
                                     \end{tabular}} (q0)
        (q0) edge [bend right] node [swap] {\tripletrans{1 \to R}
                                                        {1 \to R}
                                                        {\trans{\spc}{0}{R}}} (q1)
        (q1) edge [bend right] node [swap] {\begin{tabular}{c}
                                              \tripletrans{0,\spc \to R}
                                                          {0 \to R}
                                                          {\trans{\spc}{0}{R}} \\
                                              \tripletrans{\spc \to R}
                                                          {0 \to R}
                                                          {\trans{\spc}{0}{R}}
                                            \end{tabular}} (q0)
        (q1) edge [loop right] node {\begin{tabular}{c}
                                       \tripletrans{1 \to R}
                                                   {1 \to R}
                                                   {\trans{\spc}{1}{R}} \\
                                       \tripletrans{1 \to R}
                                                   {0,\spc \to R}
                                                   {\trans{\spc}{0}{R}} \\
                                       \tripletrans{0,\spc \to R}
                                                   {1 \to R}
                                                   {\trans{\spc}{0}{R}}
                                     \end{tabular}} (q1)
        (q0) edge [bend right, near end] node [swap] {\tripletrans{\spc \to S}
                                                     {\spc \to S}
                                                     {\spc \to S}} (accept)
        (q1) edge [bend left, near end] node {\tripletrans{\spc \to S}
                                                          {\spc \to S}
                                                          {\trans{\spc}{1}{S}}} (accept)
        ;
    \end{centerautomata}

    Tilassa $q_1$ summassa on ylivuotobittinä 1 ja tilassa $q_0$ ylivuotobittinä on 0.
  \end{enumerate}

\item 
  Esitä tilakaaviona kolminauhainen Turing-kone, mikä saa yhdellä nauhalla
  syötteenä binääriluvun ja kirjoittaa toiselle nauhalle binäärilukua
  vastaavan määrän kirjainta $a$. Kolmatta nauhaa voit käyttää jos tarvitset
  (onkohan tää liian vaikea...).

\item
  Esitä tilakaaviona kolminauhainen Turing-kone, mikä saa kahdellä
  ensimmäisellä nauhalla yhden binääriluvun kullakin, ja joka laskee kolmannen
  nauhan avulla syötelukujen kertolaskun (onkohan tää liian vaikea...).
\end{enumerate}

%% \vfill

%% \centerline{\bf Tehtävän~1 kaaviot seuraavalla sivulla!}

%% \newpage

%% Tehtävän 1.(a) deterministinen Turingin kone:
%% \begin{center}
%% \input{pics/TMesim02.pspdftex}
%% \end{center}
%% Tehtävän 1.(b) epädeterministinen Turingin kone:
%% \begin{center}
%% \input{pics/NDTMesim01.pspdftex}
%% \end{center}


\end{document}
